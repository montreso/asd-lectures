\title[ASD - Analisi di algoritmi]{\textbf{Algoritmi e Strutture Dati}\\[18pt]Analisi di algoritmi\\Funzioni di costo, notazione asintotica}


%-------------------------------------------------------------------------
\FrameTitle{}

% %-------------------------------------------------------------------------
% \begin{PlainFrame}{Sommario}
%  \tableofcontents
% \end{PlainFrame}

%%%%%%%%%%%%%%%%%%%%%%%%%%%%%%%%%%%%%%%%%%%%%%%%%%%%%%%%%%%%%%%%%%%%%%%%%%
\section{Introduzione all'analisi}

\subsection{Notazione asintotica}

\begin{frame}{Notazioni $O$, $\Omega$, $\Theta$}

\vspace{-9pt}
\begin{myboxtitle}[Definizione -- Notazione $O$]
Sia $g(n)$ una funzione di costo; indichiamo con $O(g(n))$ l'insieme
delle funzioni $f(n)$ tali per cui:\\[-6pt]
\[
  \exists c>0, \exists m \geq 0: \alert{f(n) \leq cg(n)}, \forall n \geq m
\]
\end{myboxtitle}

\medskip
\begin{itemize}
\item Come si legge: $f(n)$ è “\alert{O grande}” (big-O) di $g(n)$
\item Come si scrive: $f(n) = O(g(n))$
\item $g(n)$ è un \alert{limite asintotico superiore} per $f(n)$
\item $f(n)$ cresce al più come $g(n)$
\end{itemize}

\end{frame}

\begin{frame}{Notazioni $O$, $\Omega$, $\Theta$}

\vspace{-9pt}
\begin{myboxtitle}[Definizione -- Notazione $\Omega$]
Sia $g(n)$ una funzione di costo; indichiamo con $\Omega(g(n))$ l'insieme
delle funzioni $f(n)$ tali per cui:\\[-6pt]
\[
  \exists c>0, \exists m \geq 0: \alert{f(n) \geq cg(n)}, \forall n \geq m
\]
\end{myboxtitle}

\medskip
\begin{itemize}
\item Come si legge: $f(n)$ è “\alert{Omega grande}”  di $g(n)$
\item Come si scrive: $f(n) = \Omega(g(n))$
\item $g(n)$ è un \alert{limite asintotico inferiore} per $f(n)$
\item $f(n)$ cresce almeno quanto $g(n)$
\end{itemize}
\end{frame}

\begin{frame}{Notazioni $O$, $\Omega$, $\Theta$}

\vspace{-9pt}
\begin{myboxtitle}[Definizione -- Notazione $\Theta$]
Sia $g(n)$ una funzione di costo; indichiamo con $\Theta(g(n))$ l'insieme
delle funzioni $f(n)$ tali per cui:\\[-6pt]
\[
  \exists c_1>0, \exists c_2>0, \exists m \geq 0: \alert{c_1g(n) \leq f(n) \leq c_2g(n)}, \forall n \geq m
\]
\end{myboxtitle}

\medskip
\begin{itemize}
\item Come si legge: $f(n)$ è “\alert{Theta}” di $g(n)$
\item Come si scrive: $f(n) = \Theta(g(n))$
\item $f(n)$ cresce esattamente come $g(n)$
\item $f(n) = \Theta(g(n))$ se e solo se $f(n) = O(g(n))$ e $f(n) = \Omega(g(n))$
\end{itemize}

\end{frame}

