\title[ASD - Analisi di algoritmi]{\textbf{Algoritmi e Strutture Dati}\\[18pt]Analisi di algoritmi\\Ricorrenze, metodo dell'albero di ricorsione}

%-------------------------------------------------------------------------
\FrameTitle{}


%%%%%%%%%%%%%%%%%%%%%%%%%%%%%%%%%%%%%%%%%%%%%%%%%%%%%%%%%%%%%%%%%%%%%%%%%%
\section{Ricorrenze}

\subsection{Introduzione}

\begin{frame}{Introduzione}

\begin{overprint}

\onslide<1|handout:1>
\begin{myboxtitle}[Equazioni di ricorrenza]
Quando si calcola la complessità di un algoritmo ricorsivo, questa viene espressa tramite un'\alert{equazione di ricorrenza}, ovvero una formula matematica definita in  maniera... ricorsiva!
\end{myboxtitle}

\onslide<2-3|handout:2-3>
\begin{myboxtitle}[Forma chiusa]
Il nostro obiettivo è ottenere, quando possibile, una \alert{formula chiusa} che rappresenti la classe di complessità della funzione.
\end{myboxtitle}

\end{overprint}

\bigskip
\begin{overprint}
\onslide<1-2|handout:1-2>
\begin{myboxtitle}[MergeSort]
\[
T(n) = 
  \begin{cases}
      T(\lfloor n/2 \rfloor)  +  T(\lceil n/2 \rceil)  + \Theta(n) & n > 1 \\
     \Theta(1) & n \leq 1
  \end{cases} 
\]	
\end{myboxtitle}

\onslide<3|handout:3>
\begin{myboxtitle}[MergeSort]
\[
T(n) = \Theta(n \log n)
\]
\end{myboxtitle}

\end{overprint}

\end{frame}

\begin{frame}{Oltre l'analisi di algoritmi}

\begin{mybox}
Utilizzeremo le equazioni di ricorrenza anche per risolvere problemi
\end{mybox}

\begin{myboxtitle}[Problema]
Un bambino scende una scala composta da $n$ scalini. Ad ogni passo, può decidere di fare $1$,$2$,$3$,$4$ scalini alla volta. Determinare in quanti modi diversi può scendere le scale. Ad esempio, se $n=7$, alcuni dei modi possibili sono i seguenti:
\BI
\item 1,1,1,1,1,1,1
\item 1,2,4
\item 4,2,1
\item 2,2,2,1
\item 1,2,2,1,1
\EI
\end{myboxtitle}


\end{frame}

\begin{frame}{Oltre l'analisi di algoritmi}

\begin{myboxtitle}[Soluzione]
Sia $M(n)$ il numero di modi in cui è possibile scegliere $n$ scalini; allora $M(n)$ può essere espresso nel modo seguente:
\[
M(n) = \begin{cases}
  0 & n < 0 \\
  1 & n = 0 \\
  \sum_{k = 1}^{4} M(n-k) & n>0
\end{cases}
\]
\end{myboxtitle}

\begin{mybox}
Questa ricorrenza può essere trasformata in un algoritmo tramite semplice ricorsione o tramite programmazione dinamica.
\end{mybox}


\begin{myboxtitle}[Numeri di Tetranacci]
1, 1, 2, 4, 8, 15, 29, 56, 108, 208, 401, 773, 1490, 2872, 5536, \ldots
\end{myboxtitle}


\end{frame}


\subsection{Albero di ricorsione, o per livelli}

\begin{frame}{Metodo dell'albero di ricorsione, o per livelli}

\begin{myboxtitle}[Metodi per risolvere ricorrenze]
\begin{itemize}
\item \alert{Analisi per livelli}
\item Analisi per tentativi, o per sostituzione
\item Metodo dell'esperto, o delle ricorrenze comuni
\end{itemize}
\end{myboxtitle}


\begin{myboxtitle}[Metodo dell'albero di ricorsione, o per livelli]
“Srotoliamo” la ricorrenza in un albero i cui nodi rappresentano i costi ai vari livelli della ricorsione
\end{myboxtitle}

\end{frame}

\begin{frame}{Primo esempio}

\vspace{-6pt}
\begin{mybox}
\[
T(n) = 
  \begin{cases}
     T(n/2) + b & n > 1 \\
     c & n \leq 1
  \end{cases} 
\]
\end{mybox}

\EE possibile risolvere questa ricorrenza nel modo seguente:

\begin{columns}
\begin{column}{0.54\textwidth}
\begin{eqnarray*}
T(n) &=& b + T(n/2) \\
     &=& b + b + T(n/4) \\
     &=& b + b + b + T(n/8) \\
     &=& \ldots \\
     &=& \underbrace{b + b + \ldots + b }_{\log n}+ T(1)
\end{eqnarray*}

\end{column}
\begin{column}{0.45\textwidth}
\begin{mybox}
Assumiamo per semplicità:\\
$n=2^k$, ovvero $k = \log n$
\end{mybox}	
\onslide<2>
\begin{mybox}
$T(n) = b \log n + c = \Theta(\log n)$	
\end{mybox}	
\end{column}
\end{columns}


\end{frame}

\begin{frame}{Secondo esempio}

\vspace{-6pt}
\begin{mybox}
\[
T(n) = \begin{cases}
     4 T(n/2) + n & n > 1 \\
     1 & n \leq 1
  \end{cases}
\]
\end{mybox}

\EE possibile risolvere questa ricorrenza nel modo seguente:
\begin{overprint}

%%%%%%%%%%%%%%%%%%%
\onslide<1|handout:1>
\begin{align*}
T(n) &= n + 4T(n/2) \\
     &= n + 4n/2 + 16 T(n/2^2) \\
     &= n + 2n + 16n/4 + 64 T(n/8) \\
     &= \ldots \\
     &= n + 2n + 4n + 8 n + \ldots + 2^{\log n-1}n + 4^{\log n} T(1) \\
     &= n \sum_{j=0}^{\log n-1} 2^j + 4^{\log n}
\end{align*}

%%%%%%%%%%%%%%%%%%%
\onslide<2|handout:2>
\begin{flalign*}
T(n) = n \sum_{j=0}^{\log n-1} 2^j \phantom{\quad n \cdot \frac{2^{\log n}-1}{2-1} \quad  n(n-1)}+ 4^{\log n} &&
\end{flalign*}


%%%%%%%%%%%%%%%%%%%
\onslide<3|handout:3>
\begin{flalign*}
T(n) = \cancel{n \sum_{j=0}^{\log n-1} 2^j} \quad \alert{n \cdot \frac{2^{\log n}-1}{2-1}} \phantom{\quad  n(n-1)}+ 4^{\log n} &&
\end{flalign*}
\begin{mybox}
Serie geometrica finita:
\[
  \forall x \neq 1: \sum_{j=0}^k x^j = \frac{x^{k+1}-1}{x-1} 
  %\Rightarrow \sum_{j=0}^{63} 2^j = \frac{2^{63+1}-1}{2-1} = 2^{64} -1
\]
\end{mybox}

%%%%%%%%%%%%%%%%%%%
\onslide<4|handout:4>
\begin{flalign*}
T(n) &= \cancel{n \sum_{j=0}^{\log n-1} 2^j} \quad \cancel{n \cdot \frac{2^{\log n}-1}{2-1}} \quad \alert{n(n-1)} + 4^{\log n} &&
\end{flalign*}

\begin{mybox}
Passaggi algebrici
\end{mybox}

%%%%%%%%%%%%%%%%%%%
\onslide<5|handout:5>
\begin{flalign*}
T(n) &= \cancel{n \sum_{j=0}^{\log n-1} 2^j} \quad \cancel{n \cdot \frac{2^{\log n}-1}{2-1}} \quad n(n-1) + \cancel{4^{\log n}} \quad \alert{n^{\log 4}} &&
\end{flalign*}

\begin{mybox}
Cambiamento di base:
\begin{align*}	
\log_b n = (\log_b a) \cdot (\log_a n) & \Rightarrow \\
   a^{\log_b n} = n^{\log_b a} &
\end{align*}
\end{mybox}

%%%%%%%%%%%%%%%%%%%
\onslide<6|handout:6>
\begin{flalign*}
T(n) &= \cancel{n \sum_{j=0}^{\log n-1} 2^j} \quad \cancel{n \cdot \frac{2^{\log n}-1}{2-1}} \quad n(n-1) + \cancel{4^{\log n}} \quad \cancel{n^{\log 4}} \quad n^2 &&
\end{flalign*}

\medskip
\begin{flalign*}
\phantom{T(n)} &= 2n^2-n = \Theta(n^2)&&
\end{flalign*}


\end{overprint}

\end{frame}

\begin{frame}[shrink=5]{Terzo esempio}

\vspace{-6pt}
\begin{mybox}
\[
T(n) = \begin{cases}
     4T(n/2) + n^3 & n > 1 \\
     1 & n \leq 1
  \end{cases}
\]
\end{mybox}

Proviamo a visualizzare l'albero delle chiamate, per i primi tre livelli:

{\tiny
\begin{eqnarray*}
& n^3 &\\
&\overbrace{\mleft(\frac{n}{2}\mright)^3 \hspace{78pt} \mleft(\frac{n}{2}\mright)^3 \hspace{78pt} \mleft(\frac{n}{2}\mright)^3 \hspace{78pt} \mleft(\frac{n}{2}\mright)^3} & \\
& \overbrace{\mleft(\frac{n}{4}\mright)^3 \, \mleft(\frac{n}{4}\mright)^3 \, \mleft(\frac{n}{4}\mright)^3 \, \mleft(\frac{n}{4}\mright)^3} 
  \overbrace{\mleft(\frac{n}{4}\mright)^3 \, \mleft(\frac{n}{4}\mright)^3 \, \mleft(\frac{n}{4}\mright)^3 \, \mleft(\frac{n}{4}\mright)^3} 
  \overbrace{\mleft(\frac{n}{4}\mright)^3 \, \mleft(\frac{n}{4}\mright)^3 \, \mleft(\frac{n}{4}\mright)^3 \, \mleft(\frac{n}{4}\mright)^3} 
  \overbrace{\mleft(\frac{n}{4}\mright)^3 \, \mleft(\frac{n}{4}\mright)^3 \, \mleft(\frac{n}{4}\mright)^3 \, \mleft(\frac{n}{4}\mright)^3} &
\end{eqnarray*}
}

\end{frame}

\begin{frame}[noframenumbering,shrink=5]{Terzo esempio}

\vspace{-6pt}
\begin{mybox}
\[
T(n) = \left\{ 
  \begin{array}{ll}
     4T(n/2) + n^3 & n > 1 \\
     1 & n \leq 1
  \end{array} 
\right.
\]
\end{mybox}

\begin{center}
\begin{tabular}{|c|c|c|c|c|}
\hline
\textbf{Livello} & \textbf{Dim.} & \textbf{Costo chiam.} & \textbf{N. chiamate} & \textbf{Costo livello} \\\hline
$0$ & $n$   & $n^3$     & $1$  & $n^3$ \\\hline
$1$ & $n/2$ & $(n/2)^3$ & $4$  & $4(n/2)^3$\\\hline
$2$ & $n/4$ & $(n/4)^3$ & $16$ & $16(n/4)^3$\\\hline
$\cdots$ & $\cdots$ & $\cdots$ & $\cdots$ & $\cdots$ \\\hline
$i$ & $n/2^i$ & $(n/2^i)^3$ & $4^i$ & $4^i (n/2^i)^3$ \\\hline
$\cdots$ & $\cdots$ & $\cdots$ & $\cdots$ & $\cdots$ \\\hline
${\ell-1}$ & $n/2^{\ell-1}$ & $(n/2^{\ell-1})^3$ & $4^{\ell-1}$ & $4^{\ell-1} (n/2^{\ell-1})^3$ \\\hline
$\ell = \log n$ & 1 & $T(1)$ & $4^{\log n}$ & $4^{\log n}$ \\\hline
\end{tabular}
\end{center}

\end{frame}

\begin{frame}[noframenumbering,shrink=5]{Terzo esempio}

\vspace{-6pt}
\begin{mybox}
\[
T(n) = \left\{ 
  \begin{array}{ll}
     4T(n/2) + n^3 & n > 1 \\
     1 & n \leq 1
  \end{array} 
\right.
\]
\end{mybox}

La sommatoria dà origine a:
\begin{overprint}
\onslide<1|handout:1>
\TwoColsCustom{0.43}{0.55}{
\begin{flalign*}
  T(n) &= \sum_{i=0}^{\log n-1} 4^i \cdot n^3 / 2^{3i} \ + 4^{\log n} &&\\ 
       &= \alert{n^3} \sum_{i=0}^{\log n-1} \alert{\frac{2^{2i}}{2^{3i}}} \ + 4^{\log n} && \\
       &= n^3 \sum_{i=0}^{\log n-1} \alert{\mleft(\frac{1}{2}\mright)^i} \ + 4^{\log n} && 
\end{flalign*}
}{
\vspace{42pt}
\begin{mybox}
Passaggi algebrici
\end{mybox}
\bigskip
\begin{mybox}
Passaggi algebrici
\end{mybox}
}
\onslide<2|handout:2>
\TwoColsCustom{0.43}{0.55}{
\begin{flalign*}
T(n) &= n^3 \sum_{i=0}^{\log n-1} \mleft(\frac{1}{2}\mright)^i \ + 4^{\log n} && \\
     &= n^3 \sum_{i=0}^{\log n-1} \mleft(\frac{1}{2}\mright)^i \ + \alert{n^2} && \\
     &\alert{\leq} n^3 \sum_{i=0}^{\alert{\infty}} \mleft(\frac{1}{2}\mright)^i \ + n^2 && \\
\end{flalign*}
}{
\vspace{42pt}
\begin{mybox}
Cambiamento di base
\end{mybox}
\bigskip
\begin{mybox}
Estensione della sommatoria
\end{mybox}
}
\onslide<3|handout:3>
\TwoColsCustom{0.43}{0.55}{
\begin{flalign*}
T(n) &\leq n^3 \sum_{i=0}^{\infty} \mleft(\frac{1}{2}\mright)^i \ + n^2 && \\
     &= n^3 \cdot \alert{\frac{1}{1-\frac{1}{2}}}    + n^2&& \\
     &= \alert{2n^3} + n^2 && \\
\end{flalign*}
}{
\vspace{30pt}
\begin{mybox}
Serie geometrica infinita decrescente:
\[
 \forall x, |x|<1: \sum_{i=0}^{\infty} x^i = \frac{1}{1-x}
\]
\end{mybox}
}
\end{overprint}

\end{frame}

\begin{frame}[noframenumbering,shrink=5]{Terzo esempio}

\vspace{-6pt}
\begin{mybox}
\[
T(n) = \begin{cases}
     4T(n/2) + n^3 & n > 1 \\
     1 & n \leq 1
  \end{cases} 
\]
\end{mybox}

Abbiamo dimostrato che:
\[
  T(n) \leq 2n^3 + n^2
\]

\bigskip
\BIL
\item Possiamo affermare che  $T(n) = O(n^3)$
\item La dimostrazione precedente non afferma che $T(n)=\Theta(n^3)$, perché ad un certo punto siamo passati a $\leq$
\item Però è possibile notare che $T(n) \geq n^3$, quindi è possibile affermare che
$T(n)=\Omega(n^3)$ e quindi $T(n)=\Theta(n^3)$
\EIL

\end{frame}

\begin{frame}[shrink=10]{Quarto esempio}

\vspace{-6pt}
\begin{mybox}
\[
T(n) = \begin{cases}
     4T(n/2) + n^2 & n > 1 \\
     1 & n \leq 1
  \end{cases}
\]
\end{mybox}

\begin{overprint}

\onslide<1|handout:1>
\begin{center}
\begin{tabular}{|c|c|c|c|c|}
\hline
\textbf{Livello} & \textbf{Dimensione} & \textbf{Costo chiamata} & \textbf{N. chiamate} & \textbf{Costo livello} \\\hline
$0$ & $n$   & $n^2$     & $1$  & $n^2$ \\\hline
$1$ & $n/2$ & $(n/2)^2$ & $4$  & $4(n/2)^2$\\\hline
$2$ & $n/4$ & $(n/4)^2$ & $16$ & $16(n/4)^2$\\\hline
$\cdots$ & $\cdots$ & $\cdots$ & $\cdots$ & $\cdots$ \\\hline
$i$ & $n/2^i$ & $(n/2^i)^2$ & $4^i$ & $4^i (n/2^i)^2$ \\\hline
$\cdots$ & $\cdots$ & $\cdots$ & $\cdots$ & $\cdots$ \\\hline
${\ell-1}$ & $n/2^{\ell-1}$ & $(n/2^{\ell-1})^2$ & $4^{\ell-1}$ & $4^{\ell-1} (n/2^{\ell-1})^2$ \\\hline
$\ell = \log n$ & 1 & $T(1)$ & $4^{\log n}$ & $4^{\log n}$ \\\hline
\end{tabular}
\end{center}

\onslide<2|handout:2>
\begin{align*}
  T(n) &= \sum_{i=0}^{\log n-1} n^2 / 2^{2i} \cdot 4^i + 4^{\log n} \\
       &= n^2 \sum_{i=0}^{\log n-1} \frac{2^{2i}}{2^{2i}} + n^2 \\
       &= n^2 \sum_{i=0}^{\log n-1} 1 + n^2 \\
       &= n^2 \log n + n^2 = \Theta(n^2 \log n)
\end{align*}

\end{overprint}
\end{frame}

