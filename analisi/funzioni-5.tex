\title[ASD - Analisi di algoritmi]{\textbf{Algoritmi e Strutture Dati}\\[18pt]Analisi di algoritmi\\Ricorrenze comuni}

%-------------------------------------------------------------------------
\FrameTitle{}


%%%%%%%%%%%%%%%%%%%%%%%%%%%%%%%%%%%%%%%%%%%%%%%%%%%%%%%%%%%%%%%%%%%%%%%%%%
\subsection{Metodo dell'esperto}


\begin{frame}{Metodo dell'esperto}

\begin{myboxtitle}[Metodi per risolvere ricorrenze]
\begin{itemize}
\item Metodo dell'albero di ricorsione, o per livelli
\item Metodo di sostituzione, o per tentativi
\item \alert{Metodo dell'esperto, o delle ricorrenze comuni}
\end{itemize}
\end{myboxtitle}


\begin{myboxtitle}[Ricorrenze comuni]
Esiste un'ampia classe di ricorrenze che possono essere risolte facilmente 
facendo ricorso ad alcuni teoremi, ognuno dei quali si occupa di una classe
particolare di equazioni di ricorrenza.	
\end{myboxtitle}

\end{frame}

\begin{frame}{Ricorrenze lineari con partizione bilanciata}

\begin{myboxtitle}[Teorema]
Siano $a$ e $b$ costanti intere tali che $a \ge 1$ e $b \ge 2$, e $c$,
$\beta$ costanti reali tali che $c > 0$ e $\beta \ge 0$. Sia
$T(n)$ data dalla relazione di ricorrenza:

\[
T(n) = \begin{cases}
  aT(n/b) + cn^\beta & n>1 \\
  d & n \leq 1
\end{cases}
\]

Posto $\alpha = \log a / \log b = \log_b a$, allora:

\[
T(n) =\begin{cases}
\Theta(n^\alpha) & \alpha > \beta \\
\Theta(n^\alpha\log n) & \alpha = \beta \\
\Theta(n^\beta) & \alpha < \beta 
\end{cases}
\]
\end{myboxtitle}	
	
\end{frame}

\begin{frame}{Ricorrenze lineari con partizione bilanciata}

\begin{myboxtitle}[Assunzioni]
Assumiamo che $n$ sia una potenza intera di $b$: $n=b^k, k= \log_b n$
\end{myboxtitle}

\begin{myboxtitle}[Perchè ci serve?]
Semplifica tutti i calcoli successivi
\end{myboxtitle}

\begin{myboxtitle}[Influisce sul risultato?]
\BI
\item Supponiamo che l'input abbia dimensione $b^k+1$
\item Estendiamo l'input fino ad una dimensione $b^{k+1}$ (\alert{padding})
\item L'input è stato esteso al massimo di un fattore costante $b$
\item Ininfluente al fine della complessità computazionale
\EI
\end{myboxtitle}

\end{frame}


\begin{frame}{Ricorrenze lineari con partizione bilanciata}

\vspace{-6pt}
\begin{mybox}
$T(n) = aT(n/b) + cn^\beta \qquad T(1) = d$
\end{mybox}	
\bgroup
\begin{center}
\def\arraystretch{1.1}
\begin{tabular}{|c|c|c|c|c|}
\hline
\textbf{Liv.} & \textbf{Dim.} & \textbf{Costo chiam.} & \textbf{N. chiamate} & \textbf{Costo livello} \\\hline
$0$ & $b^k$ & $cb^{k\beta}$  & $1$ & $cb^{k\beta}$ \\
\hline
$1$ & $b^{k-1}$ & $cb^{(k-1)\beta}$ & $a$ & $acb^{(k-1)\beta}$ \\
\hline
$2$ & $b^{k-2}$ & $cb^{(k-2)\beta}$& $a^2$  & $a^2cb^{(k-2)\beta}$ \\
\hline
$\cdots$ & $\cdots$ & $\cdots$ & $\cdots$ & $\cdots$ \\
\hline
$i$  & $b^{k-i}$ & $cb^{(k-i)\beta}$ & $a^i$& $a^icb^{(k-i)\beta}$ \\
\hline
$\cdots$ & $\cdots$ & $\cdots$ & $\cdots$ & $\cdots$ \\
\hline
$k-1$  & $b$ & $cb^\beta$ & $a^{k-1}$ & $a^{k-1}cb^{\beta}$ \\
\hline
$k$ & $1$ & $d$ & $a^k$ &$da^k$ \\
\hline
\end{tabular}
\end{center}
\egroup

\end{frame}

\begin{frame}{Ricorrenze lineari con partizione bilanciata}

\bgroup
\begin{center}
\def\arraystretch{1.1}
\begin{tabular}{|c|c|c|c|c|}
\hline
\textbf{Liv.} & \textbf{Dim.} & \textbf{Costo chiam.} & \textbf{N. chiamate} & \textbf{Costo livello} \\\hline
$i$  & $b^{k-i}$ & $cb^{(k-i)\beta}$ & $a^i$& $a^icb^{(k-i)\beta}$ \\
\hline
$k$ & $1$ & $d$ & $a^k$ &$da^k$ \\
\hline
\end{tabular}
\end{center}
\egroup

\begin{mybox}
Sommando i costi totali di tutti i livelli, si ottiene:

\[
T(n) =\quad da^k + cb^{k\beta} \sum_{i=0}^{k-1} \frac{a^i}{b^{i\beta}} =\quad da^k + cb^{k\beta} \sum_{i=0}^{k-1} \left(\frac{a}{b^{\beta}}\right)^i
\]
\end{mybox}

\end{frame}

\begin{frame}{Ricorrenze lineari con partizione bilanciata}

\begin{mybox}
\[
T(n) =\quad da^k + cb^{k\beta} \sum_{i=0}^{k-1} \frac{a^i}{b^{i\beta}} =\quad da^k + cb^{k\beta} \sum_{i=0}^{k-1} \left(\frac{a}{b^{\beta}}\right)^i
\]
\end{mybox}

\begin{myboxtitle}[Osservazioni]
\BIL
\item $\alert{a^k} = a^{\log_b n} = a^{\log n / \log b} = 2^{\log a \log n / \log b}=
n^{\log a / \log b} = \alert{n^\alpha}$
\item $\alpha = \log a / \log b \Rightarrow \alpha \log b = \log a \Rightarrow \log b^\alpha = \log a \Rightarrow \alert{a= b^{\alpha}}$
\item Poniamo $\alert{q} = \frac{a}{b^\beta} = \frac{b^\alpha}{b^\beta} = \alert{b^{\alpha-\beta}}$
\EIL
\end{myboxtitle}

\begin{mybox}
\[
T(n) = da^k + cb^{k\beta} \sum_{i=0}^{k-1} \left(\frac{a}{b^{\beta}}\right)^i = dn^\alpha  + cb^{k\beta} \sum_{i=0}^{k-1} q^i
\]
\end{mybox}
	
\end{frame}

\begin{frame}[shrink=5]{Ricorrenze lineari con partizione bilanciata}
	
	
\BB{Caso 1: $\alpha > \beta$}	

Ne segue che: $q = b^{\alpha-\beta} > 1$:

\begin{align*}
T(n) &= dn^\alpha  + cb^{k\beta} \textstyle\sum_{i=0}^{k-1} q^i \\
&=n^\alpha d + cb^{k\beta} \alert{[(q^k - 1)/(q - 1)]} && \textrm{Serie geometrica finita} \\
&\le n^\alpha d + cb^{k\beta}\alert{q^k}/(q - 1) && \textrm{Disequazione}\\
&= n^\alpha d + \frac{cb^{k\beta}\alert{a^k}}{\alert{b^{k\beta}}}/(q - 1) && \textrm{Sostituzione $q$}\\
&= n^\alpha d + \textit{ca}^k/(q - 1) && \textrm{Passi algebrici}\\
&= \alert{n^\alpha[d+ c/(q - 1)]} && a^k = n^\alpha, \textrm{raccolta termini}
\end{align*}

\BIL
\item Quindi $T(n)$ è $O(n^\alpha)$. 
\item Per via della componente $dn^\alpha$, $T(n)$ è anche $\Omega(n^\alpha)$,
e quindi $T(n) = \Theta(n^\alpha)$.
\EIL


\end{frame}

\begin{frame}{Ricorrenze lineari con partizione bilanciata}
	
\BB{Caso 2: $\alpha = \beta$}	

Ne segue che: $q = b^{\alpha-\beta} = 1$:

\begin{align*}
T(n) &= dn^\alpha  + cb^{k\beta} \textstyle\sum_{i=0}^{k-1} q^i \\
&= n^\alpha d + cn^\beta \alert{k} && q^i = 1^i = 1\\
&= n^\alpha d + c n^{\alert{\alpha}} k && \alpha=\beta\\
&= n^\alpha (d + ck) && \textrm{Raccolta termini}\\
&= n^\alpha [d + c\alert{\log n / \log b}] && k = \log_b n
\end{align*}

e quindi $T(n)$ è $\Theta(n^\alpha \log n)$;

\end{frame}

\begin{frame}{Ricorrenze lineari con partizione bilanciata}
	
\BB{Caso 3: $\alpha < \beta$}	

Ne segue che: $q = b^{\alpha-\beta} < 1$:

\begin{align*}
T(n) &= dn^\alpha  + cb^{k\beta} \textstyle\sum_{i=0}^{k-1} q^i \\
&= n^\alpha d + cb^{k\beta} \alert{[(q^k-1)/(q-1)]} && \textrm{Serie geometrica finita}\\
&= n^\alpha d + cb^{k\beta} \alert{[(1 - q^k)/(1 - q)]} && \textrm{Inversione}\\
&\le n^\alpha d + cb^{k\beta} [\alert{1}/(1 - q)] && \textrm{Disequazione}\\
&= n^\alpha d + c\alert{n}^\beta/(1 - q) && b^k=n
\end{align*}

\BIL
\item Quindi $T(n)$ è $O(n^\beta)$. 
\item Poichè $T(n) = \Omega(n^\beta)$ per il termine non ricorsivo, si ha che
$T(n) = \Theta(n^\beta)$. 
\EIL

\end{frame}




\begin{frame}{Ricorrenze lineari con partizione bilanciata (Estesa)}

\begin{myboxtitle}[Teorema]
Sia $a \geq 1$, $b > 1$, $f(n)$ asintoticamente positiva, e sia 

\[
T(n) = \begin{cases}
     a T(n/b) + f(n) & n > 1 \\
     d & n \leq 1
  \end{cases} 
\]
Sono dati tre casi:

\vspace{-12pt}
\bgroup
\begin{center}
\def\arraystretch{1.5}
\begin{tabular}{|P{0.5cm}|P{5cm}P{0.4cm}P{3.7cm}|}
\hline
(1) & $\exists \epsilon>0: f(n) = O(n^{\log_b a - \epsilon})$ & $\Rightarrow$ & $T(n) = \Theta(n^{\log_b a})$\\
\hline
(2) & $f(n) = \Theta(n^{\log_b a})$ & $\Rightarrow$ & $T(n) = \Theta(f(n) \log n)$ \\
\hline
~\newline(3) & $\exists \epsilon>0: f(n) = \Omega(n^{\log_b a + \epsilon}) \wedge {}$\newline
$\exists c: 0 < c < 1, \exists m \geq 0:$  \newline
$af(n/b) \leq cf(n), \forall n \geq m$
& ~\newline $\Rightarrow$ 
& ~\newline $T(n) = \Theta(f(n))$\\
\hline
\end{tabular}
\end{center}
\egroup

\end{myboxtitle}

\end{frame}


\begin{frame}{Alcuni esempi}

\bgroup
\def\arraystretch{1.1}

\begin{overprint}
\onslide<1|handout:0>
\begin{tabular}{|c|c|c|c|c|c|}
\hline
\textbf{Ricorrenza} & $\mathbf{a}$ & $\mathbf{b}$ & $\mathbf{\textbf{log}_b a}$ & \textbf{Caso} & \textbf{Funzione} \\
\hline
$T(n) = 9T(n/3)+n \log n$ &  &  &  &  &  \\
\hline
\end{tabular}
\onslide<2|handout:1>
\begin{tabular}{|c|c|c|c|c|c|}
\hline
\textbf{Ricorrenza} & $\mathbf{a}$ & $\mathbf{b}$ & $\mathbf{\textbf{log}_b a}$ & \textbf{Caso} & \textbf{Funzione} \\
\hline
$T(n) = 9T(n/3)+n \log n$ & 9 & 3 & 2 & (1) & $T(n) = \Theta(n^2)$ \\
\hline
\end{tabular}
\end{overprint}
\egroup

\onslide<2|handout:1>
\begin{mybox}
$f(n) = n \log n = O(n^{\log_b a - \epsilon}) = O(n^{2-\epsilon})$, con $\epsilon < 1$
\end{mybox}

\end{frame}

\begin{frame}{Alcuni esempi}

\bgroup
\def\arraystretch{1.1}
\begin{overprint}

\onslide<1|handout:0>
\begin{tabular}{|c|c|c|c|c|c|}
\hline
\textbf{Ricorrenza} & $\mathbf{a}$ & $\mathbf{b}$ & $\mathbf{\textbf{log}_b a}$ & \textbf{Caso} & \textbf{Funzione} \\
\hline
$T(n) = T(2n/3)+1$ & &  & &  &  \\
\hline
\end{tabular}

\onslide<2|handout:1>
\begin{tabular}{|c|c|c|c|c|c|}
\hline
\textbf{Ricorrenza} & $\mathbf{a}$ & $\mathbf{b}$ & $\mathbf{\textbf{log}_b a}$ & \textbf{Caso} & \textbf{Funzione} \\
\hline
$T(n) = T(2n/3)+1$ & 1 & $\frac{3}{2}$ & 0 & (2) & $T(n) = \Theta(\log n)$ \\
\hline
\end{tabular}

\end{overprint}
\egroup

\onslide<2|handout:1>
\begin{mybox}
$f(n) = n^0 = \Theta(n^{\log_b a}) = \Theta(n^0)$ 
\end{mybox}

\end{frame}

\begin{frame}{Alcuni esempi}

\bgroup
\def\arraystretch{1.1}
\begin{overprint}

\onslide<1|handout:0>

\begin{tabular}{|c|c|c|c|c|c|}
\hline
\textbf{Ricorrenza} & $\mathbf{a}$ & $\mathbf{b}$ & $\mathbf{\textbf{log}_b a}$ & \textbf{Caso} & \textbf{Funzione} \\
\hline
$T(n) = 3T(n/4)+n \log n$ &  &  &  & &  \\
\hline
\end{tabular}

\onslide<2|handout:1>

\begin{tabular}{|c|c|c|c|c|c|}
\hline
\textbf{Ricorrenza} & $\mathbf{a}$ & $\mathbf{b}$ & $\mathbf{\textbf{log}_b a}$ & \textbf{Caso} & \textbf{Funzione} \\
\hline
$T(n) = 3T(n/4)+n \log n$ & 3 & 4 & $\approx 0.79$ & (3) & $T(n) = \Theta(n \log n)$ \\
\hline
\end{tabular}

\end{overprint}

\egroup

\onslide<2|handout:1>
\begin{mybox}
 $f(n) = n \log n = \Omega(n^{\log_4 3 + \epsilon})$, con $\epsilon < 1-\log_4 3 \approx 0.208$ \\[3pt]
  
Dobbiamo dimostrare che: 

$\exists c\leq 1, \exists m \geq 0: af(n/b) \leq cf(n), \forall n \geq m$

\begin{align*}
af(n/b) &= 3 (n/4 \log n/4) \\
&= 3/4 n (\log n - \log 4) \\
&\leq 3/4 n \log n \\
&\stackrel{?}{\leq} c n \log n
\end{align*}

L'ultima disequazione è soddisfatta da $c=3/4$ e qualsiasi $m$.

\end{mybox}

\end{frame}



\begin{frame}[shrink=6]{Alcuni esempi}

\bgroup
\def\arraystretch{1.1}

\begin{overprint}

\onslide<1|handout:0>

\begin{tabular}{|c|c|c|c|c|c|}
\hline
\textbf{Ricorrenza} & $\mathbf{a}$ & $\mathbf{b}$ & $\mathbf{\textbf{log}_b a}$ & \textbf{Caso} & \textbf{Funzione} \\
\hline
$T(n) = 2T(n/2)+n \log n$ &  &  &  &  &  \\
\hline
\end{tabular}

\onslide<2|handout:1>

\begin{tabular}{|c|c|c|c|c|c|}
\hline
\textbf{Ricorrenza} & $\mathbf{a}$ & $\mathbf{b}$ & $\mathbf{\textbf{log}_b a}$ & \textbf{Caso} & \textbf{Funzione} \\
\hline
$T(n) = 2T(n/2)+n \log n$ & 2 & 2 & 1 & -- & Non applicabile \\
\hline
\end{tabular}

\end{overprint}

\egroup

\onslide<2|handout:1>
\begin{mybox}
$f(n) = n \log n \neq O(n^{1-\epsilon})$, con $\epsilon>0$\\[3pt]
$f(n) = n \log n \neq \Theta(n)$\\[3pt]
$f(n) = n \log n \neq \Omega(n^{1+\epsilon})$, con $\epsilon>0$\\[3pt]
   
Nessuno dei tre casi è applicabile e bisogna utilizzare altri metodi.
\end{mybox}

\end{frame}

\begin{frame}{Interi inferiori/superiori}

\BIL
\item I teoremi appena visti non considerano equazioni di ricorrenza con interi inferiori/superiori

\[
T(n) = 2T(n/2) + n
\]

\item In realtà, le equazioni di ricorrenza dovrebbero sempre essere espresse tramite interi inferiori/superiori, perché operano su dimensioni dell'input

\[
T(n) = T(\lfloor n/2 \rfloor) + T(\lceil n/2 \rceil) + n
\]

\item I risultati dei teoremi valgono anche quando le funzioni sono espresse tramite interi inferiori/superiori

\EIL
	
\end{frame}


\begin{frame}{Ricorrenze lineari di ordine costante}

\begin{myboxtitle}[Teorema]
Siano $a_1$, $a_2$, \ldots, $a_h$ costanti intere non negative, con $h$
costante positiva, $c$ e $\beta$ costanti reali tali che $c > 0$ e $\beta \ge
0$, e sia $T(n)$ definita dalla relazione di ricorrenza:

\[
T(n) = \begin{cases}
\sum_{1 \le i \le h} a_iT(n - i) + cn^\beta & n > m \\
\Theta(1) & n \le m \le h
\end{cases}
\]

Posto $a = \displaystyle\sum_{1 \le i \le h}a_i$, allora:
\begin{enumerate}
\item $T(n)$ è $\Theta(n^{\beta+1})$, se $a = 1$,
\item $T(n)$ è $\Theta(a^nn^\beta)$, se $a \ge 2$.
\end{enumerate}

\end{myboxtitle}

\end{frame}


\begin{frame}[shrink=6]{Alcuni esempi}

\bgroup
\def\arraystretch{1.1}

\begin{overprint}

\onslide<1|handout:0>

\begin{tabular}{|c|c|c|c|c|c|c|}
\hline
&\textbf{Ricorrenza} & $\mathbf{a}$ & $\mathbf{\beta}$ & \textbf{Caso} & \textbf{Funzione} \\
\hline
(A) & $T(n) = T(n-10)+n^2$ &  &  &  &  \\
\hline
(B) & $T(n) = T(n-2)+T(n-1)+1$ &  &  & &  \\
\hline
\end{tabular}

\onslide<2|handout:1>

\begin{tabular}{|c|c|c|c|c|c|c|}
\hline
&\textbf{Ricorrenza} & $\mathbf{a}$ & $\mathbf{\beta}$ & \textbf{Caso} & \textbf{Funzione} \\
\hline
(A) & $T(n) = T(n-10)+n^2$ & 1 & 2 & (1) & $T(n) = \Theta(n^3)$ \\
\hline
(B) & $T(n) = T(n-2)+T(n-1)+1$ & 2 & 0 & (2) & $T(n) = 2^n$ \\
\hline
\end{tabular}

\end{overprint}

\egroup

\onslide<2|handout:1>
\begin{mybox}
\BIL
\item[(A)] Poiché $a=1$, il costo è polinomiale.
\item[(B)] Poiché $a=2$, il costo è esponenziale.
\EIL
\end{mybox}

\end{frame}

\begin{frame}{Esercizio}
	
Siano
\BI
\item $T(n) = 7T(n/2) + n^2$ una funzione di costo di un algoritmo $A$, e
\item $T'(n) = aT'(n/4) + n^2$ una funzione di costo di un algoritmo $A'$.
\EI
Qual è il massimo valore intero di $a$ che rende $A'$ asintoticamente più veloce
di $A$?
	
\end{frame}

\begin{frame}{Esercizio -- Soluzione}
	
\BB{
Poichè $\log_2 7$ è $\approx 2.81$, il Master Theorem dice che
$T(n) = \Theta(n^{\log_2 7})$.
}

\BB{
Utilizzando alcune trasformazioni algebriche, si ottiene che:
\begin{align*}
\log_2 7 &= \frac{\log_4 7}{\log_4 2} 
         = \frac{\log_4 7}{1/2} \\
         &= 2\log_4 7
         = \log_4 49 
\end{align*}
}

\BB{
\BI
\item $a<16 \Rightarrow \alpha = \log_4 a < 2 \Rightarrow T'(n) = \Theta(n^2) = O(T(n))$
\item $a=16 \Rightarrow \alpha = \log_4 a = 2 \Rightarrow T'(n) = \Theta(n^2 \log n) = O(T(n))$
\item $16<a \leq 49 \Rightarrow \alpha = \log_4 a > 2 \Rightarrow T'(n) = \Theta(n^\alpha) = O(T(n))$
\item $a < 49 \Rightarrow \alpha = \log_4 a > 2 \Rightarrow T'(n) = \Theta(n^\alpha) = \Omega(T(n))$
\EI
}
	
\end{frame}



