\input templates/header
\title[ASD - Ricerca locale]{\textbf{Algoritmi e Strutture Dati}\\[24pt]Ricerca locale}

\usepackage{xcolor}
\usepackage{colortbl}
\usepackage{epigraph}
\usepackage{tikz}
\usetikzlibrary{trees}
\usetikzlibrary{matrix}
\usetikzlibrary{graphs}
\usetikzlibrary{shapes}
\usetikzlibrary{positioning}
\usepackage{xmpmulti}
\usepackage{listings}

\lstset{
  basicstyle=\ttfamily,
  columns=fullflexible,
  keywordstyle=\color{red}\bfseries,
  commentstyle=\color{blue},
  showstringspaces=false,
}

\tikzset{
    Node/.style = {circle, draw=black, align=center, fill=yellow!40, thick}
}    
\tikzset{
    Edge/.style = {draw=black,thick,-latex}
}    

\newcommand*\circled[1]{\tikz[baseline=(char.base)]{
            \node[circle,ball color=blue, shade, 
 color=white,inner sep=1.2pt] (char) {\tiny #1};}}

\newcommand{\R}[1]{\textcolor{red}{#1}}
\newcommand{\B}[1]{\textcolor{blue}{#1}}
\newcommand{\G}[1]{\textcolor{violet}{#1}}

\graphicspath{{figs/15/}}

\begin{document}


%-------------------------------------------------------------------------
\FrameTitle{}

%-------------------------------------------------------------------------
\FrameContent



%%%%%%%%%%%%%%%%%%%%%%%%%%%%%%%%%%%%%%%%%%%%%%%%%%%%%%%%%%%%%%%%%%%%%%%%%%
\section{Introduzione}

%-------------------------------------------------------------------------
\begin{frame}{Ricerca locale}

Se si conosce una soluzione ammissibile (non necessariamente ottima) ad un
problema di ottimizzazione, si può cercare una soluzione migliore
nelle "vicinanze" di quella precedente. 

\bigskip
Si continua così fino a quando non si può più migliorare


\begin{Procedure}
\caption[A]{\ricercalocale()}
$\Sol = \textrm{una soluzione ammissibile del problema}$\;
\While{$\exists S \in I(\Sol)$ migliore di \Sol}
{
  $\Sol = S$\;
}
\Return \Sol\;
\end{Procedure}


\end{frame}

\begin{frame}{Ricerca locale}

\vspace{-9pt}
\begin{center}
\IG{0.85}{search.pdf}
\end{center}

\end{frame}


%%%%%%%%%%%%%%%%%%%%%%%%%%%%%%%%%%%%%%%%%%%%%%%%%%%%%%%%%%%%%%%%%%%%%%%%%%
\section{Flusso massimo}

\subsection{Rete di flusso}

%%---------------------------------------------------------------------------
\begin{frame}{Rete di flusso, sorgente, pozzo, capacità}

\includegraphics[width=1.0\textwidth]{template.pdf}

\end{frame}


%-------------------------------------------------------------------------
\begin{frame}{Rete di flusso}

\vspace{-9pt}
\begin{myboxtitle}[Definizione]
Una \alert{rete di flusso $G=(V,E,s,t,c)$} è data da:
\BI
\item un grafo orientato $G=(V,E)$
\item un nodo $s \in V$ detto \alert{sorgente}
\item un nodo $t \in V$ detto \alert{pozzo}
\item una funzione di \alert{capacità} $c: V \times V \rightarrow \mathbb{R}^{\geq 0}$,\\
tale che $(x,y) \not \in E \Rightarrow c(x,y)=0$.
\EI
\end{myboxtitle}

\begin{myboxtitle}[Assunzioni]
\BIL
\item Per ogni nodo $x \in V$, esiste un cammino $s \leadsto x \leadsto t$ 
da $s$ a $t$ che passa per $x$.
\item Possiamo ignorare i nodi che non godono di questa proprietà
\EIL
\end{myboxtitle}

\end{frame}







\subsection{Flusso}

%%---------------------------------------------------------------------------
\begin{frame}{Flusso}

\vspace{-9pt}
\begin{myboxtitle}[Flusso]
Un \alert{flusso} in $G$ è una funzione \alert{$f: V \times V \rightarrow \mathbb{R}$} 
che soddisfa le seguenti proprietà:
\BIL
\item \alert{Vincolo sulla capacità}: $\forall x,y \in V$, $f(x,y) \leq c(x,y)$

\item \alert{Antisimmetria}: $\forall x,y \in V$, $f(x,y) = -f(y,x)$

\item \alert{Conservazione del flusso}: $\forall x \in V-\{s,t\}$, 
$\sum_{y \in V} f(x,y) = 0$
\EIL
\end{myboxtitle}

\end{frame}

%%---------------------------------------------------------------------------
\begin{frame}{Flusso}

\vspace{-9pt}
\begin{myboxtitle}[Vincolo sulla capacità]
Il flusso non deve eccedere la capacità sull'arco.

\[
\forall x,y \in V: {\color{red} f(x,y)} \leq {\color{blue} c(x,y)}
\]
\end{myboxtitle}

\smallskip
\begin{center}
\IG{0.3}{prop1.pdf}
\end{center}

\end{frame}


%%---------------------------------------------------------------------------
\begin{frame}{Flusso}

\vspace{-9pt}
\begin{myboxtitle}[Simmetria opposta]
Il flusso che attraversa un arco in direzione $(x,y)$ è l'opposto del flusso che attraversa l'arco in direzione $(y,x)$.

\[
\forall x,y \in V: {\color{red} f(x,y)} = {\color{blue} -f(y,x)}
\]
\end{myboxtitle}

\smallskip
\begin{center}
\IG{0.3}{prop2.pdf}
\end{center}

\BB{Il flusso viene definito in questo modo per semplificare la proprietà successiva e altre regole}

\end{frame}

%%---------------------------------------------------------------------------
\begin{frame}{Flusso}

\vspace{-9pt}
\begin{myboxtitle}[Conservazione del flusso]
Per ogni nodo diverso da sorgente e pozzo, la somma dei flussi entranti deve essere uguale alla somma dei flussi uscenti.

\[
\forall \textcolor{red}{x} \in V-\{s,t\}: \sum_{\textcolor{blue}{y} \in V} f(\textcolor{red}{x}, \textcolor{blue}{y}) = 0
\]
\end{myboxtitle}

\smallskip
\begin{center}
\IG{0.4}{prop3.pdf}
\end{center}

\end{frame}

%%---------------------------------------------------------------------------
\begin{frame}{Definizioni}

\vspace{-9pt}
\begin{myboxtitle}[Valore del flusso]
Il \alert{valore di un flusso} $f$ è definito come:

\[
  |f| = \sum_{(s, x) \in E} f(s, x)
\]

ovvero come la quantità di flusso uscente da $s$.
\end{myboxtitle}

\begin{columns}[T]
\column{0.3\textwidth}
\[|f| = 3\]
\column{0.75\textwidth}
\vspace{-12pt}
\IG{0.7}{valoreflusso.pdf}
\end{columns}

\end{frame}

\subsection{Problema}

%%---------------------------------------------------------------------------
\begin{frame}{Problema}

\vspace{-9pt}
\begin{myboxtitle}[Flusso massimo]
Data una rete $G=(V,E,s,t,c)$, trovare un flusso che abbia valore massimo fra 
tutti i flussi associabili alla rete.

\[
  |f^*| = \max \{ |f| \}
\]
\end{myboxtitle}

\begin{columns}[T]
\column{0.3\textwidth}
\[|f^*| = 4\]
\column{0.75\textwidth}
\vspace{-12pt}
\IG{0.7}{flussomassimo.pdf}
\end{columns}

\end{frame}

%%---------------------------------------------------------------------------
\begin{frame}{Flusso massimo nella guerra fredda}

\TwoColsCustom{0.52}{0.46}{
\vspace{-12pt}
\IG{1.0}{russian.pdf}
}{
\footnotesize
\href{https://web.eecs.umich.edu/~pettie/matching/Harris-Ross-fundamentals-of-evaluating-rail-net-capacities-RAND-report-declassified.pdf}{T. Harris, F. Ross, Fundamentals of a Method for Evaluating Rail Net Capacities. Research
Memorandum RM-1573, The RAND Corporation, Santa Monica, California, 1955}

\bigskip
Rete ferroviaria dell'Unione Sovietica occidentale e dell'Europa Orientale. $|V| = 44, |E|=105$. \\
Flusso massimo di $163 \cdot 10^6$ kg. \\
}

\bigskip
\BB{
Alexander Schrijver. \href{https://homepages.cwi.nl/~lex/files/histtrpclean.pdf}{On the history of the transportation and
maximum flow problems}
}
\end{frame}

%%---------------------------------------------------------------------------
\begin{frame}{Esercizio - Trovate il flusso massimo}
\vspace{-12pt}
\IG{0.9}{flow.pdf}
\tiny
\url{https://disi.unitn.it/~montreso/asd/material/flow/flusso-esempio1.pdf}
\end{frame}

\subsection{Metodo delle reti residue}


%%---------------------------------------------------------------------------
\begin{frame}{Metodo delle reti residue}

\vspace{-9pt}
\begin{myboxtitle}[Algoritmo, informale]
\BIL
\item Si memorizza un flusso "corrente" $f$, inizialmente nullo
\item Si ripetono le operazioni seguenti:
  \BIL
  \item Si "sottrae" il flusso attuale dalla rete iniziale, ottenendo una
    rete residua
  \item Si cerca un flusso $g$ all'interno della rete residua
  \item Si somma $g$ ad $f$
  \EIL
  fino a quando non è più possibile trovare un flusso positivo $g$
\EIL
\end{myboxtitle}

\begin{myboxtitle}[Output]
\EE possibile dimostrare che questo approccio restituisce
un flusso massimo
\end{myboxtitle}

\end{frame}



%%---------------------------------------------------------------------------
\begin{frame}{Definizioni}

\vspace{-9pt}
\begin{myboxtitle}[Flusso nullo]
Definiamo \alert{flusso nullo} la funzione 
$f_0: V \times V \rightarrow \mathbf{R}^{\geq 0}$ tale che:

\[
  \forall x,y \in V: f(x,y) = 0
\]
\end{myboxtitle}

\bigskip
\begin{columns}[T]
\column{0.3\textwidth}
\[|f| = 0\]
\column{0.75\textwidth}
\vspace{-12pt}
\IG{0.7}{flussonullo.pdf}
\end{columns}

\end{frame}


%-------------------------------------------------------------------------
\begin{frame}{Definizioni}

\vspace{-9pt}
\begin{myboxtitle}[Somma di flussi]
Per ogni coppia di flussi $f_1$ e $f_2$ in $G$, definiamo il \alert{flusso somma} $g = f_1+f_2$ come un flusso tale che 

\[
  \forall x,y \in V: g(x,y) = f_1(x,y) + f_2(x,y)
\]
\end{myboxtitle}

\IG{0.7}{sum.pdf}

\end{frame}

%%---------------------------------------------------------------------------
\begin{frame}{Definizioni}

\vspace{-9pt}
\begin{myboxtitle}[Capacità residua]
Definiamo \alert{capacità residua} di un flusso $f$ in una rete $G=(V,E,s,t,c)$ 
una funzione $c_f: V \times V \rightarrow \mathbf{R}^{\geq 0}$
tale che 

\[
  \forall x,y \in V: c_f(x,y) = c(x,y) - f(x,y)
\]
\end{myboxtitle}

\smallskip
\begin{center}
\IG{0.4}{prop4.pdf}
\end{center}

\BI
\item Flusso in \R{rosso}
\item Capacità residua in \G{viola}
\item Capacità iniziale in \B{blu}
\EI
\end{frame}

%%---------------------------------------------------------------------------
\begin{frame}{Definizioni}

\vspace{-9pt}
\begin{myboxtitle}[Capacità residua]
Definiamo \alert{capacità residua} di un flusso $f$ in una rete $G=(V,E,s,t,c)$ 
una funzione $c_f: V \times V \rightarrow \mathbf{R}^{\geq 0}$
tale che 

\[
  \forall x,y \in V: c_f(x,y) = c(x,y) - f(x,y)
\]
\end{myboxtitle}

\smallskip
\begin{center}
\IG{0.4}{prop5.pdf}
\end{center}

Per la definizione di capacità residua, si creano degli archi all'indietro:
\begin{align*}
  c_f(b,a) &= c(b,a) - f(b,a) \\
           &= 0 - (-5)  = 5 \\
\end{align*}

\end{frame}

%%---------------------------------------------------------------------------
\begin{frame}{Definizioni}

\vspace{-9pt}
\begin{myboxtitle}[Reti residue]
Data una rete di flusso $G=(V,E,s,t,c)$ e un flusso $f$ su $G$, 
possiamo costruire una \alert{rete residua} $G_f=(V,E_f,s,t,c_f)$, tale
per cui:

\[
  \forall x,y \in V: (x,y) \in E_f \Leftrightarrow c_f(x,y) > 0
\]
\end{myboxtitle}

\smallskip
\begin{center}
\IG{0.7}{reteresidua.pdf}
\end{center}

\end{frame}

\subsection{Algoritmo}

%%---------------------------------------------------------------------------
\begin{frame}{Algoritmo, schema generale}

\begin{Procedure}
\caption[A]{$\INTARRAY[\,]$ \Flusso(\Graph $G$, \Node $s$, \Node $t$, $\INTARRAY[\,]\ c$)}
$f = f_0$\REMR{Inizializza un flusso nullo}
$c_f = c$\REMR{Capacità iniziale}
\Repeat{$g = f_0$}{
  $g = \textrm{trova un flusso in $c_f$ tale che $|g| > 0$, altrimenti $f_0$}$\;
  $f = f + g$\;
  $c_f = \textrm{Capacità residua del flusso $f$ in $G$}$\;
}
\Return $f$
\end{Procedure}

\end{frame}

%%---------------------------------------------------------------------------
\begin{frame}{Dimostrazione correttezza}

\vspace{-9pt}
\begin{myboxtitle}[Lemma]
Se $f$ è un flusso in $G$ e $g$ è un flusso in $G_f$, allora
$f+g$ è un flusso in $G$.
\end{myboxtitle}

\begin{overprint}
\onslide<1|handout:0>
\TwoColsCustom{0.35}{0.62}{
\begin{myboxtitle}[Dimostrazione?]
\BI
\item Vincolo sulla capacità?
\item Antisimmetria?
\item Conservazione del flusso?
\EI
\end{myboxtitle}
}{
\begin{myboxtitle}[Reminder]
\begin{align*}
& G_f  =(V,E_f,s,t,c_f)\\
& \forall x,y \in V: (x,y) \in E_f \Leftrightarrow c_f(x,y) > 0 \\
& \forall x,y \in V: c_f(x,y) = c(x,y) - f(x,y)
\end{align*}
\end{myboxtitle}
}
\onslide<2|handout:1>
\begin{myboxtitle}[Conservazione ($\forall x \in V - \{ s,t \}$)]
\small
\begin{align*}
  \sum_{y \in V} (f+g)(x,y) &= \sum_{y \in V} (f(x,y) + g(x,y)) \\
  &= \sum_{y \in V} f(x,y) + \sum_{y \in V} g(x,y) \\
   &= 0 + 0
\end{align*}
\end{myboxtitle}
\onslide<3|handout:2>
\begin{myboxtitle}[Antisimmetria ($\forall x,y \in V$)]
\small
\begin{align*}
f(x,y)+g(x,y) &= -f(y,x)-g(y,x) & \textrm{\small Antisimmetria $f$, $g$} \\
f(x,y)+g(x,y) &= -(f(y,x) + g(y,x)) & \textrm{\small Raccolta segno $-$} \\
(f+g)(x,y) &= -(f+g)(y,x) & \textrm{\small Sostituzione} 
\end{align*}
\end{myboxtitle}
\onslide<4|handout:3>
\begin{myboxtitle}[Vincolo sulla capacità ($\forall x,y \in V$)]
\small
\begin{align*}
g(x,y) &\leq c_f(x,y) & \textrm{\small $g$ è un flusso in $G_f$} \\
\alert{f(x,y)} + g(x,y) &\leq  \textcolor{blue}{c_f(x,y}) \qquad\quad~~ + \alert{f(x,y)} & \textrm{\small Aggiungo termine uguale}\\
(f+g)(x,y) &\leq \textcolor{blue}{c(x,y)-f(x,y)} + f(x,y) & \textrm{\small Sostituzione}\\
(f+g)(x,y) &\leq c(x,y) & \textrm{\small Semplificazione}
\end{align*}
\end{myboxtitle}

\end{overprint}

\end{frame}

%%---------------------------------------------------------------------------
\begin{frame}{Flusso aggiuntivo}

\vspace{-9pt}
\begin{myboxtitle}[Domanda]
Il problema principale del metodo proposto è il seguente:\\ 
\alert{Come trovare un 
flusso aggiuntivo}? 
\end{myboxtitle}

\begin{center}
\IG{0.8}{template.pdf}
\end{center}

\end{frame}



%%---------------------------------------------------------------------------
\begin{frame}{Cammini Aumentanti: Ford-Fulkerson, 1956}

\vspace{-9pt}
\begin{overprint}
\onslide<1|handout:1>
\BIL
\item Si trova un cammino $C=v_0,v_1,\ldots, v_n$, con $s=v_0$ e $t=v_n$ 
  nella rete residua $G_f$; 
\item Si identifica la \alert{capacità del cammino}, corrispondente
  alla minore capacità degli archi incontrati (\alert{collo di bottiglia}):
  \smallskip
  \[
    c_f(C) = \min_{i=2 \ldots n} c_f(v_{i-1},v_i)
  \]
\EIL
\onslide<2-|handout:2->
\BIL
\item si crea un flusso addizionale $g$ tale che 
  \BI
  \item $g(v_{i-1}, v_i) = c_f(C)$;
  \item $g(v_i,v_{i-1}) = -c_f(C)$ (per antisimmetria)
  \item $g(x,y)=0$ per tutte le altre coppie $(x,y)$
  \EI
\EIL
\end{overprint}

\medskip
\begin{overprint}
\onslide<1|handout:1>
\begin{center}
\IG{0.7}{cammino-aumentante.pdf}
\end{center}
\onslide<2|handout:2>
\begin{center}
\IG{0.7}{flusso-aumentante-errato1.pdf}
\end{center}
\onslide<3|handout:3>
\begin{center}
\IG{0.7}{flusso-aumentante-errato2.pdf}
\end{center}
\onslide<4|handout:4>
\begin{center}
\IG{0.7}{flusso-aumentante-errato3.pdf}
\end{center}
\onslide<5|handout:5>
\begin{center}
\IG{0.7}{flusso-aumentante.pdf}
\end{center}
\end{overprint}



\end{frame}

%%---------------------------------------------------------------------------
\begin{frame}{Cammini Aumentanti: Ford-Fulkerson, 1956}

\vspace{-9pt}
\begin{Procedure}
\caption[A]{$\INTARRAY[\,]$ \Flusso(\Graph $G$, \Node $s$, \Node $t$, $\INTARRAY[\,]\ c$)}
$\INTARRAY[\,]\ f = \NEW\ \INTARRAY[\,]$\REMR{Flusso parziale}
$\INTARRAY[\,]\ g = \NEW\ \INTARRAY[\,]$\REMR{Flusso da cammino aumentante}
$\INTARRAY[\,]\ c_f = \NEW\ \INTARRAY[\,]$\REMR{Rete residua}
\BlankLine
\ForEach{$x,y \in G.\VV()$}{
  $f[x][y] = 0$\REMR{Inizializza un flusso nullo}
  $c_f[x][y] = c[x][y]$\REMR{Copia c in r}
}
\Repeat{$g=f_0$}{
  $g = \textrm{flusso associato ad un cammino aumentante in $r$, oppure $f_0$}$\;
  \ForEach{$x,y \in G.\VV()$}{
    $f[x][y] = f[x][y] + g[x][y]$\REMR{$f = f + g$}
    $c_f[x][y] = c[x][y]	 - f[x][y]$\REMR{Calcola $c_f$}
  }  
}
\Return $f$
\end{Procedure}

\end{frame}

%%---------------------------------------------------------------------------
\begin{frame}{Esecuzione}

\begin{columns}[T]
\column{0.75\textwidth}
\begin{overprint}
\includegraphics<1|handout:1>[width=\textwidth,page=1]{esempio.pdf}
\includegraphics<2|handout:2>[width=\textwidth,page=2]{esempio.pdf}
\includegraphics<3|handout:0>[width=\textwidth,page=3]{esempio.pdf}
\includegraphics<4|handout:0>[width=\textwidth,page=4]{esempio.pdf}
\includegraphics<5|handout:0>[width=\textwidth,page=5]{esempio.pdf}
\includegraphics<0|handout:3>[width=\textwidth,page=17]{esempio.pdf}
\includegraphics<6|handout:4>[width=\textwidth,page=6]{esempio.pdf}
\includegraphics<7|handout:0>[width=\textwidth,page=7]{esempio.pdf}
\includegraphics<8|handout:0>[width=\textwidth,page=8]{esempio.pdf}
\includegraphics<9|handout:0>[width=\textwidth,page=9]{esempio.pdf}
\includegraphics<0|handout:5>[width=\textwidth,page=18]{esempio.pdf}
\includegraphics<10|handout:6>[width=\textwidth,page=10]{esempio.pdf}
\includegraphics<11|handout:0>[width=\textwidth,page=11]{esempio.pdf}
\includegraphics<12|handout:0>[width=\textwidth,page=12]{esempio.pdf}
\includegraphics<13|handout:0>[width=\textwidth,page=13]{esempio.pdf}
\includegraphics<14|handout:0>[width=\textwidth,page=14]{esempio.pdf}
\includegraphics<15|handout:0>[width=\textwidth,page=15]{esempio.pdf}
\includegraphics<0|handout:7>[width=\textwidth,page=16]{esempio.pdf}
\end{overprint}
\column{0.20\textwidth}
\begin{overprint}
\onslide<3|handout:0>
\B{$f(\mathtt{a},\mathtt{b}) = 2$}\\
\onslide<4|handout:0>
$f(\mathtt{a},\mathtt{b}) = 2$\\
\B{$f(\mathtt{b},\mathtt{e}) = 2$}\\
\onslide<5|handout:0>
$f(\mathtt{a},\mathtt{b}) = 2$\\
$f(\mathtt{b},\mathtt{e}) = 2$\\
\B{$f(\mathtt{e},\mathtt{f}) = 2$}\\
\onslide<6|handout:3-4>
$f(\mathtt{a},\mathtt{b}) = 2$\\
$f(\mathtt{b},\mathtt{e}) = 2$\\
$f(\mathtt{e},\mathtt{f}) = 2$\\
\onslide<7|handout:0>
$f(\mathtt{a},\mathtt{b}) = 2$\\
$f(\mathtt{b},\mathtt{e}) = 2$\\
$f(\mathtt{e},\mathtt{f}) = 2$\\
\B{$f(\mathtt{a},\mathtt{c}) = 2$}\\
\onslide<8|handout:0>
$f(\mathtt{a},\mathtt{b}) = 2$\\
$f(\mathtt{b},\mathtt{e}) = 2$\\
$f(\mathtt{e},\mathtt{f}) = 2$\\
$f(\mathtt{a},\mathtt{c}) = 2$\\
\B{$f(\mathtt{c},\mathtt{e}) = 2$}\\
\onslide<9|handout:0>
$f(\mathtt{a},\mathtt{b}) = 2$\\
$f(\mathtt{b},\mathtt{e}) = 2$\\
\B{$f(\mathtt{e},\mathtt{f}) = \xout{2} \ 4$}\\
$f(\mathtt{a},\mathtt{c}) = 2$\\
$f(\mathtt{c},\mathtt{e}) = 2$\\
\onslide<10|handout:5-6>
$f(\mathtt{a},\mathtt{b}) = 2$\\
$f(\mathtt{b},\mathtt{e}) = 2$\\
$f(\mathtt{e},\mathtt{f}) = \xout{2} \ 4$\\
$f(\mathtt{a},\mathtt{c}) = 2$\\
$f(\mathtt{c},\mathtt{e}) = 2$\\
\onslide<11|handout:0>
$f(\mathtt{a},\mathtt{b}) = 2$\\
$f(\mathtt{b},\mathtt{e}) = 2$\\
$f(\mathtt{e},\mathtt{f}) = \xout{2} \ 4$\\
\B{$f(\mathtt{a},\mathtt{c}) = \xout{2} \ 4$}\\
$f(\mathtt{c},\mathtt{e}) = 2$\\
\onslide<12|handout:0>
$f(\mathtt{a},\mathtt{b}) = 2$\\
$f(\mathtt{b},\mathtt{e}) = 2$\\
$f(\mathtt{e},\mathtt{f}) = \xout{2} \ 4$\\
$f(\mathtt{a},\mathtt{c}) = \xout{2} \ 4$\\
\B{$f(\mathtt{c},\mathtt{e}) = \xout{2} \ 4$}\\
\onslide<13|handout:0>
$f(\mathtt{a},\mathtt{b}) = 2$\\
\B{$f(\mathtt{b},\mathtt{e}) = \xout{2} \ 0$}\\
$f(\mathtt{e},\mathtt{f}) = \xout{2} \ 4$\\
$f(\mathtt{a},\mathtt{c}) = \xout{2} \ 4$\\
$f(\mathtt{c},\mathtt{e}) = \xout{2} \ 4$\\
\onslide<14|handout:0>
$f(\mathtt{a},\mathtt{b}) = 2$\\
$f(\mathtt{b},\mathtt{e}) = \xout{2} \ 0$\\
$f(\mathtt{e},\mathtt{f}) = \xout{2} \ 4$\\
$f(\mathtt{a},\mathtt{c}) = \xout{2} \ 4$\\
$f(\mathtt{c},\mathtt{e}) = \xout{2} \ 4$\\
\B{$f(\mathtt{b},\mathtt{d}) = 2$}\\
\onslide<15|handout:0>
$f(\mathtt{a},\mathtt{b}) = 2$\\
$f(\mathtt{b},\mathtt{e}) = \xout{2} \ 0$\\
$f(\mathtt{e},\mathtt{f}) = \xout{2} \ 4$\\
$f(\mathtt{a},\mathtt{c}) = \xout{2} \ 4$\\
$f(\mathtt{c},\mathtt{e}) = \xout{2} \ 4$\\
$f(\mathtt{b},\mathtt{d}) = 2$\\
\B{$f(\mathtt{d},\mathtt{f}) = 2$}\\
\onslide<0|handout:7>
$f(\mathtt{a},\mathtt{b}) = 2$\\
$f(\mathtt{b},\mathtt{e}) = \xout{2} \ 0$\\
$f(\mathtt{e},\mathtt{f}) = \xout{2} \ 4$\\
$f(\mathtt{a},\mathtt{c}) = \xout{2} \ 4$\\
$f(\mathtt{c},\mathtt{e}) = \xout{2} \ 4$\\
$f(\mathtt{b},\mathtt{d}) = 2$\\
$f(\mathtt{d},\mathtt{f}) = 2$\\
\end{overprint}

\end{columns}

\end{frame}


%%---------------------------------------------------------------------------
\begin{frame}{Ricerca del cammino}

Ford e Fulkerson (1956) parlavano genericamente di una visita per trovare un cammino.

\bigskip
Edmonds e Karp (1972) suggerirono di utilizzare una visita in ampiezza.

\bigskip
Costo della visita: $O(V+E)$


\end{frame}

%%---------------------------------------------------------------------------
\begin{frame}[fragile]{Ricerca del cammino}

\vspace{-18pt}
\footnotesize
\begin{lstlisting}[language=java]    
/**
 * Compute the max-flow using the Ford-Fulkerson algorithm
 * @param C the capacity matrix
 * @param s the source node
 * @param t the sink node
 * @return the flow matrix 
 */
private static int[][] flow(int[][] C, int s, int t) {
  // Create an empty flow
  int[][] F = new int[C.length][C.length];
  // Visited array to perform DFS, initially empty
  boolean[] visited = new boolean[C.length];
  // Repeat until there is no path    
  while (dfs(C, F, s, t, visited, Integer.MAX_VALUE) > 0) {
    Arrays.fill(visited, false);
  }
  return F;
}
\end{lstlisting}

\end{frame}

%%---------------------------------------------------------------------------
\begin{frame}[fragile,shrink=5]{Ricerca del cammino}

\vspace{-18pt}
\footnotesize
\begin{lstlisting}[language=java]    
/**
 * Performs a DFS starting from node i and trying to reach node t. 
 * @param C the capacity matrix; if capacity[x][y]>0, there is a edge from x to y
 * @param F the flow matrix to be computed
 * @param i the current node, 
 * @param t the sink node
 * @param visited the boolean set containing the nodes that have been visited
 * @param min the smallest capacity found during the visit.
 * @returns the value of the additional flow found during the DFS
 */
private static int dfs(int[][] C, int[][] F, int i, int t, boolean[] visited, int min) {
  if (i==t) return min;             // If sink has been reached, terminate
  visited[i] = true;
  for (int j=0; j < C.length; j++) {
    if (C[i][j] > 0 && !visited[j]) {    // Non-visited neighbor
      int val = dfs(C, F, j, t, visited, Math.min(min, C[i][j]));
      if (val > 0) {
        C[i][j] = C[i][j]-val; C[j][i] = C[j][i]+val;
        F[i][j] = F[i][j]+val; F[j][i] = F[j][i]-val;
        return val;
      }
    }
  }  
  return 0;                 // The sink has not been found
}
\end{lstlisting}

\end{frame}







%%%%%%%%%%%%%%%%%%%%%%%%%%%%%%%%%%%%%%%%%%%%%%%%%%%%%%%%%%%%%%%%%%%%%%%%%%%%%
\section{Dimostrazione di correttezza}

%%---------------------------------------------------------------------------
\begin{frame}{Dimostrazione correttezza -- Definizioni}

\vspace{-12pt}
\begin{myboxtitle}[Taglio]
Un \alert{taglio} $(S,T)$ della rete di flusso 
$G=(V,E,s,t,c)$ è una partizione di $V$ in due sottoinsiemi disgiunti $S,T$ tali che:
\smallskip
\begin{eqnarray*}
    S = V-T \\
    s \in S \wedge t \in T
\end{eqnarray*}
\end{myboxtitle}

\vspace{-12pt}
\begin{columns}[T]
\column{0.70\textwidth}
\IG{1.0}{taglio.pdf}
\column{0.28\textwidth}
\vspace{18pt}
$S = \{ \mathtt{a}, \mathtt{b}, \mathtt{c}, \mathtt{d} \}$\\
$T = \{ \mathtt{e}, \mathtt{f} \}$\\
\end{columns}


\end{frame}


%%---------------------------------------------------------------------------
\begin{frame}{Dimostrazione correttezza -- Definizioni}

\vspace{-12pt}
\begin{myboxtitle}[Capacità di un taglio]
La \alert{capacità} $C(S,T)$ attraverso il taglio $(S,T)$ è pari a:

\[
  C(S,T) = \sum_{x \in S, y \in T} c(x,y)
\]
\end{myboxtitle}

\vspace{-12pt}
\begin{columns}[T]
\column{0.70\textwidth}
\IG{1.0}{taglio2.pdf}
\column{0.28\textwidth}
\vspace{18pt}
$S = \{ \mathtt{a}, \mathtt{b}, \mathtt{c}, \mathtt{d} \}$\\
$T = \{ \mathtt{e}, \mathtt{f} \}$\\
~\\
$C(S,T) = 14$
\end{columns}

\end{frame}

%%---------------------------------------------------------------------------
\begin{frame}{Dimostrazione correttezza -- Definizioni}

\vspace{-9pt}
\begin{myboxtitle}[Flusso di un taglio]
Se $f$ è un flusso in $G$, il \alert{flusso netto} $F_f(S,T)$ 
attraverso $(S,T)$ è:

\[
  F_f(S,T) = \sum_{x \in S, y \in T} f(x,y)
\]
\end{myboxtitle}

\vspace{-12pt}
\begin{columns}[T]
\column{0.70\textwidth}
\IG{1.0}{taglio3.pdf}
\column{0.28\textwidth}
\vspace{18pt}
$S = \{ \mathtt{a}, \mathtt{b}, \mathtt{c}, \mathtt{d} \}$\\
$T = \{ \mathtt{e}, \mathtt{f} \}$\\
~\\
$C(S,T) = 14$\\
~\\
$F_f(S,T) = 6$
\end{columns}

\end{frame}

%%---------------------------------------------------------------------------
\begin{frame}{Dimostrazione correttezza -- Lemma valore del flusso}
    
\vspace{-3pt}
\begin{myboxtitle}[Lemma -- Valore del flusso di un taglio]
Dato un flusso $f$ e un taglio $(S,T$), la quantità di flusso
$F_f(S,T)$ che attraversa il taglio è uguale a $|f|$.
\end{myboxtitle}

\small
\begin{overprint}
\onslide<1|handout:0>
\IG{0.8}{taglio3.5.pdf}
\onslide<2|handout:1>
\IG{0.8}{taglio4.pdf}
\onslide<3|handout:2>
\begin{align*}
F_f(S,T) &= \sum_{x \in S, y \in T} f(x,y) \\
       &= \sum_{x \in S, y \in V} f(x,y) - \sum_{x \in S, y \in S} f(x,y)& T = V - S\\ 
       &= \sum_{x \in S, y \in V} f(x,y) - 0& \textrm{Antisimmetria}\\
       &= [...] 
\end{align*}
\onslide<4|handout:3>
\begin{align*}
F_f(S,T) &= \sum_{x \in S, y \in V} f(x,y)& \\ 
       &= \sum_{x \in S-\{s\}, y \in V} f(x,y) + \sum_{y \in V} f(s,y) & s \in S\\ 
       &= \sum_{x \in S-\{s\}} \sum_{y \in V} f(x,y) + \sum_{y \in V} f(s,y) & \mathrm{Sommatoria}\\ 
       &= \sum_{x \in S-\{s\}} 0 + \sum_{y \in V} f(s,y) & \textrm{Conservazione flusso}\\ 
       &= \sum_{y \in V} f(s,y) = |f| & \textrm{Definizione valore flusso}
\end{align*}
\end{overprint}
        
\end{frame}




%%---------------------------------------------------------------------------
\begin{frame}{Dimostrazione correttezza -- Lemma capacità del taglio}

\vspace{-3pt}
\begin{myboxtitle}[Lemma -- Capacità taglio]
Il \alert{flusso massimo} è limitato superiormente dalla capacità del \alert{taglio minimo}, ovvero il taglio la cui capacità è minore fra tutti i tagli.
\end{myboxtitle}

\begin{overprint}
\onslide<1|handout:1>
\begin{center}
\includegraphics[width=0.8\textwidth,page=1]{esempio.pdf}
\end{center}
\vspace{-12pt}
\BB{Qual è il taglio minimo?}
\onslide<2|handout:2>
\IG{0.8}{minmax.pdf}
\onslide<3|handout:3>

\medskip
\BIL
\item Nessun flusso attraverso un taglio supera la capacità del taglio
\smallskip
\[
 \forall f: F_f(S,T) \leq C(S,T) \qquad \forall (S,T)\ \textrm{taglio di $V$}
\]
Dimostrazione:
\smallskip
\[
  \forall f: F_f(S,T) = \sum_{x \in S, y \in T} f(x,y) \leq \sum_{x \in S, y \in T} c(x,y) = C(S,T)
\]
(Vincolo sulla capacità)
\EIL
\onslide<4|handout:4>

\medskip
\BIL
\item Nessun flusso attraverso un taglio supera la capacità del taglio
\smallskip
\[
  \forall f:F_f(S,T) \leq C(S,T) \qquad \forall (S,T)\ \textrm{taglio di $V$}
\]
\item Il flusso che attraversa un taglio è uguale al valore del flusso
\smallskip
\[
  \forall f:|f| = F_f(S,T) \qquad \forall (S,T)\ \textrm{taglio di $V$}
\]
\item Quindi, il valore del flusso è limitato superiormente dalla capacità
di tutti i possibili tagli. 
\smallskip
\[
\forall f: |f| \leq C(S,T) \qquad \forall (S,T)\ \textrm{taglio di $V$}
\]
\EIL
\end{overprint}

\end{frame}

%%---------------------------------------------------------------------------
\begin{frame}{Teorema del taglio minimo / flusso massimo}

\vspace{-9pt}
\begin{myboxtitle}[Teorema]
Le seguenti tre affermazioni sono equivalenti:
\begin{enumerate}
\item $f$ è un \alert{flusso massimo}
\item non esiste alcun cammino aumentante nella rete residua $G_f$
\item esiste un \alert{taglio minimo} $(S,T)$ tale che $C(S,T) = |f|$
\end{enumerate}
\end{myboxtitle}

\bigskip
Dimostreremo circolarmente:
\BIL
\item $(1) \Rightarrow (2)$
\item $(2) \Rightarrow (3)$
\item $(3) \Rightarrow (1)$
\EIL

\end{frame}

%-------------------------------------------------------------------------
\begin{frame}{Dimostrazione correttezza -- $(1) \Rightarrow (2)$}

\BB{$f$ è un flusso massimo $\Rightarrow$\\ non esiste nessun cammino aumentante nella rete residua $G_f$}

\BIL
\item Se esistesse un cammino aumentante, il flusso potrebbe essere aumentato e quindi non sarebbe massimo (assurdo).
\EIL

\end{frame}

%-------------------------------------------------------------------------
\begin{frame}{Dimostrazione correttezza -- $(2) \Rightarrow (3)$}

\vspace{-9pt}
\BB{Non esiste nessun cammino aumentante nella rete residua $G_f$ $\Rightarrow$\\
esiste un taglio minimo $(S,T)$ tale che $C(S,T) = |f|$}


\begin{center}
\IG{0.6}{dim1.pdf}
\end{center}

\small
\BI
\item Poiché non esiste nessun cammino aumentante nella rete residua $G_f$, non esiste nessun cammino
da $s$ a $t$
\item Sia $S$ l'insieme dei vertici raggiungibili da $s$; $T=V-S$
\item Ovviamente $s \in S$ e $t \in T$, quindi $(S,T)$ è un taglio
\EI

\end{frame}


%-------------------------------------------------------------------------
\begin{frame}{Dimostrazione correttezza -- $(2) \Rightarrow (3)$}

\vspace{-9pt}
\BB{Non esiste nessun cammino aumentante nella rete residua $G_f$ $\Rightarrow$\\
esiste un taglio minimo $(S,T)$ tale che $C(S,T) = |f|$}

\begin{center}
\IG{0.6}{dim2.pdf}
\end{center}

\small
\BI
\item Poiché $t$ non è raggiungibile da $s$ in $G_f$, tutti gli
archi $(x,y)$ con $x \in S$ e $y \in T$ sono saturati;  ovvero, $f(x,y) = c(x,y)$. 
\EI

\end{frame}

%-------------------------------------------------------------------------
\begin{frame}{Dimostrazione correttezza -- $(2) \Rightarrow (3)$}

\vspace{-9pt}
\BB{Non esiste nessun cammino aumentante nella rete residua $G_f$ $\Rightarrow$\\
esiste un taglio minimo $(S,T)$ tale che $C(S,T) = |f|$}

\BIL
\item Per il Lemma -- Valore del flusso di un taglio,

\[ 
  |f| = \sum_{x \in S, y \in T} f(x,y)
\]
\item Ne segue che:

\[ 
  |f| = \sum_{x \in S, y \in T} f(x,y) = \sum_{x \in S, y \in T} c(x,y) = C(S,T)
\]

\item $(S,T)$ è minimo perchè $|f|=C(S,T)$ e per ogni taglio $(S', T')$, abbiamo che $|f| \leq C(S', T')$ (per il lemma su Capacità taglio)

\EIL

\end{frame}

%-------------------------------------------------------------------------
\begin{frame}{Dimostrazione correttezza -- $(3) \Rightarrow (1)$}

\vspace{-9pt}
\BB{Esiste un taglio $(S,T)$ tale che $C(S,T) = |f|$ $\Rightarrow$\\
$f$ è un flusso massimo}

\bigskip
Poiché per un qualsiasi flusso $f$ e un qualsiasi taglio $(S,T)$ vale la relazione $|f| \leq C(S,T)$, 
il flusso che soddisfa $|f| = C(S,T)$ deve essere massimo.

\end{frame}

\section{Complessità}


%%---------------------------------------------------------------------------
\begin{frame}{Complessità}

\vspace{-3pt}
\begin{myboxtitle}[Complessità, limite superiore -- Ford-Fulkerson]
Se le capacità sono \alert	{intere}, l'algoritmo di Ford-Fulkerson
ha complessità \alert{$O((V+E)|f^*|)$} (liste) o \alert{$O(V^2|f^*|)$} (matrice).
\end{myboxtitle}

\BIL
\item L'algoritmo parte dal flusso nullo e termina quando il valore totale
del flusso raggiunge $|f^*|$
\item Ogni incremento del flusso aumenta il flusso di almeno un'unità
\item Ogni ricerca di un cammino richiede una visita del grafo, con
costo $O(V+E)$ o $O(V^2)$; 
\item La somma dei flussi e il calcolo della rete residua può essere 
effettuato in tempo $O(V+E)$ o $O(V^2)$.
\EIL

\end{frame}

%%---------------------------------------------------------------------------
\begin{frame}{Complessità}

\vspace{-3pt}
\begin{myboxtitle}[Complessità, limite superiore -- Edmonds e Karp]
Se le capacità della rete sono \alert{intere}, l'algoritmo di Edmonds e Karp
ha complessità \alert{$O(VE^2)$} nel caso pessimo.
\end{myboxtitle}

\BB{Come si conciliano i due limiti superiori?}

\BIL
\item $O(VE^2)$ vs $O((V+E)|f^*|)$
\item Sono entrambi limiti superiori
\item Sono entrambi validi
\item Si deve quindi prendere il più basso fra i due
\EIL

\end{frame}

%-------------------------------------------------------------------------
\begin{frame}{Dimostrazione complessità}

\vspace{-9pt}
\begin{myboxtitle}[Teorema]
La complessità dell'algoritmo di Edmonds-Karp è $O(VE^2)$.
\end{myboxtitle}
\BIL
\item Vengono eseguiti $O(VE)$ aumenti di flusso, ognuno dei quali richiede
una visita in ampiezza $O(V+E)$.
\item La complessità è quindi $O(VE(V+E))$.
\item Poiché $E = \Omega(V)$ (ogni nodo diverso da sorgente, pozzo ha almeno un arco entrante e un arco uscente), possiamo semplificare scrivendo che la complessità è $O(VE^2)$
\EIL

% \begin{myboxtitle}[Lemma - Aumenti di flusso]
% Il numero totale di aumenti di flusso eseguiti dall'algoritmo
% di Edmonds e Karp è $O(VE)$.
% \end{myboxtitle}

\end{frame}

%-------------------------------------------------------------------------
\begin{frame}{Dimostrazione complessità}

\vspace{-9pt}
\begin{myboxtitle}[Lemma - Monotonia]
Sia $\delta_f(s,x)$ la distanza minima da $s$ a $x$ in una rete residua $G_f$.
Sia $f' = f+g$ un flusso nella rete iniziale, con $g$ flusso non nullo derivante
da un cammino aumentante. Allora $\delta_{f'}(s,x) \geq \delta_f(s,x)$.
\end{myboxtitle}  
\BI
  \item Quando viene aumentato il flusso, alcuni archi si  ``spengono'' (capacità residua 0)
  \item Questi archi erano utilizzati nei cammini minimi (BFS)
  \item I cammini minimi non posso diventare più corti
\EI

\begin{overprint}
\onslide<1|handout:1>
\begin{center}
\includegraphics[width=0.6\textwidth,page=1]{bfs.pdf}
\end{center}
\onslide<2|handout:2>
\begin{center}
\includegraphics[width=0.6\textwidth,page=2]{bfs.pdf}
\end{center}
\onslide<3|handout:3>
\begin{center}
\includegraphics[width=0.6\textwidth,page=3]{bfs.pdf}
\end{center}
\end{overprint}

\end{frame}

%-------------------------------------------------------------------------
\begin{frame}{Dimostrazione complessità}

\vspace{-9pt}
\begin{myboxtitle}[Lemma - Aumenti di flusso]
Il numero totale di aumenti di flusso eseguiti dall'algoritmo
di Edmonds e Karp è $O(VE)$.
\end{myboxtitle}

\BIL
\item Sia $G_f$ una rete residua
\item Sia $C$ un cammino aumentante di $G_f$. 
\item $(x,y)$ è un arco \alert{critico} (collo di bottiglia) in $C$ se 

\[
c_f(x,y) = \min_{(u,v) \in C} \{ c_f(u,v) \}
\]

\item In ogni cammino esiste almeno un arco critico
\item Una volta aggiunto il flusso associato a $C$, l'arco critico scompare dalla rete residua. 
\EIL

\end{frame}

%-------------------------------------------------------------------------
\begin{frame}{Dimostrazione complessità}

\BIL
\item Poiché i cammini aumentanti sono cammini minimi, abbiamo che:
\[
 \delta_f(s,y) = \delta_f(s,x)+1
\]
\item L'arco $(x,y)$ potrà ricomparire se e solo se il flusso lungo l'arco 
diminuirà, ovvero se $(y,x)$ appare in un cammino aumentante
\item  Sia $g$ il flusso quando questo accade; come sopra, abbiamo:
\[
 \delta_{g}(s,x) = \delta_{g}(s,y)+1
\]
\item Per Lemma (Monotonia), abbiamo anche che $\delta_g(s,y)
\geq \delta_{f}(s,y)$; quindi:
\begin{eqnarray*}
\delta_{g}(s,x) &=& \delta_{g}(s,y)+1 \\
  & \geq & \delta_f(s,y) + 1 \\
  & = & \delta_f(s,x) + 2
\end{eqnarray*}
\EIL

\end{frame}

%-------------------------------------------------------------------------
\begin{frame}{Dimostrazione complessità}

\BIL
\item Dal momento in cui un nodo è critico al momento in cui può tornare ad essere critico, il cammino minimo si è allungato almeno di due passi.
\item La lunghezza massima del cammino fino a $x$, tenuto conto che poi si deve ancora seguire l'arco $(x,y)$,
è $V-2$.
\item Quindi un arco può diventare critico al massimo $(V-2)/2 = V/2-1$ volte.
\item Poiché ci sono $O(E)$ archi che possono diventare critici $O(V)$ volte,
abbiamo che il numero massimo di flussi aumentanti è $O(VE)$. 
\EIL

\end{frame}


%%---------------------------------------------------------------------------
\begin{frame}{Complessità -- Altre versioni}

\small
\begin{tabular}{|P{2.6cm}|P{3.5cm}|P{4.1cm}|}
\hline
\textbf{Nome} & \textbf{Complessità} & \textbf{Note} \\\hline
Ford-Fulkerson & $O(E|f^*|)$ & Converge con valori razionali \\\hline
Edmonds-Karp & $O(VE^2)$ & Specializzazione basata su BFS \\\hline
Dinitz, blocking flow & $O(V^2E)$ & In alcune reti particolari, $O(\min(V^{2/3},E^{1/2})E)$ \\\hline
MPM & $O(V^3)$ & Solo su DAG \\\hline
Dinitz & $O(VE \log V)$ & Struttura dati Dynamic trees \\\hline
Goldberg e Rao & \begingroup \footnotesize $O(\min(V^{2/3},E^{1/2})E \cdot {}$ \endgroup & $C=\max_{(x,y) \in E} c(x,y)$ \\
& \begingroup \footnotesize  ${} \cdot \log (V^2/E+2) \log C)$ \endgroup & \\\hline

Orlin + King,Rao,Tarjan & $O(VE)$ & 2013\\\hline
Chen, Kyng, Liu at al. & $O(V +E^{1+o(1)})$ & "Quasi-linear", 112 pagine, 2022!\\\hline
\end{tabular}

\end{frame}

%%---------------------------------------------------------------------------
\begin{frame}{Complessità -- Letture}

\BB{
Andrew V. Goldberg, Robert E. Tarjan. \href{https://cacm.acm.org/magazines/2014/8/177011-efficient-maximum-flow-algorithms/fulltext}{Efficient Maximum Flow Algorithms}.
Communications of the ACM, 57(8):82--89. 2014.
}

\BB{
Li Chen, Rasmus Kyng, Yang P. Liu, Richard Peng, Maximilian Probst Gutenberg, Sushant Sachdeva. 
\href{https://arxiv.org/abs/2203.00671}{Maximum Flow and Minimum-Cost Flow in Almost-Linear Time}.
63rd IEEE Annual Symposium on Foundations of Computer Science, FOCS 2022, Denver, CO, USA, October 31 - November 3, 2022.
}

\end{frame}

\section{Oltre il Flusso Massimo}

%-------------------------------------------------------------------------
\begin{frame}{Applicazioni}

\BIL
\item  \alert{Maximum Flows with Edge Demands (Circulation Problem)}\\
Flusso massimo con limiti inferiori e superiori per la capacità

\item \alert{Node Supplies and Demands}\\
Il flusso può essere generato o consumato in ogni nodo

\item \alert{Min-Cost Max Flows}\\
Al flusso è associato un costo, fra due flussi massimali scegliamo quello con costo minimo

\EIL
\medskip
\BB{
Jeff Erickson. \href{https://courses.engr.illinois.edu/cs573/fa2010/notes/24-maxflowext.pdf}{\underline{Lecture 18: Extensions of Maximum Flow}} (8 pp.)
}
\BB{
David P. Williamson. \href{http://www.networkflowalgs.com/book.pdf}{\underline{Network Flow Algorithms}}.\\ 
Cambridge University Press, 2019 (273 pp.)
}


\end{frame}


%-------------------------------------------------------------------------
\begin{frame}{Applicazioni}

\vspace{-9pt}
\begin{columns}[T]
\column{0.43\textwidth}
\BIL
\item Bipartite matching
\item Data mining
\item Project selection
\item Airline scheduling
\item Baseball elimination
\item Image segmentation
\item Network connectivity
\EIL
\column{0.55\textwidth}
\BIL
\item Network reliability
\item Distributed computing
\item Egalitarian stable matching
\item Security of statistical data
\item Network intrusion detection
\item Multi-camera scene reconstruction
\item Gene function prediction
\EIL
\end{columns}

\medskip
\BB{
Jeff Erickson. \href{https://jeffe.cs.illinois.edu/teaching/algorithms/book/11-maxflowapps.pdf}{\underline{Lecture 11: Applications of Flows and Cuts}}.
}


\end{frame}

\subsection{Abbinamento grafi bipartiti}

%-------------------------------------------------------------------------
\begin{frame}{Abbinamento (matching) massimo nei grafi bipartiti}

\vspace{-9pt}
\begin{myboxtitle}[Problema - Job Assignment - Input]
\BIL
\item Un insieme $J$ contenente $n$ job
\item Un insieme $W$ contenente $m$ worker
\item Una relazione $R \subseteq J \times W$, tale che $(j,w) \in R$
se e solo se il job $j$ può essere eseguito dal worker $w$
\EIL
\end{myboxtitle}

\begin{myboxtitle}[Problema - Job Assignment - Output]
\BIL
\item Il più grande sottoinsieme $O \subseteq R$, tale che:
  \BI
  \item ogni job venga assegnato al più ad un worker
  \item ad ogni worker venga assegnato al più un job
  \EI
\EIL
\end{myboxtitle}

\end{frame}
  
%-------------------------------------------------------------------------
\begin{frame}{Esempio}

\begin{overprint}
\onslide<1|handout:1>
\begin{center}
\includegraphics[height=7cm,page=1]{bipartite.pdf}
\end{center}
\onslide<2|handout:2>
\begin{center}
\includegraphics[height=7cm,page=2]{bipartite.pdf}
\end{center}
\end{overprint}
  
\end{frame}
    
\end{document}
