\input templates/header

\usepackage{tikz}
\usepackage{xmpmulti}
\usepackage{listings}


\lstset{
  basicstyle=\ttfamily,
  columns=fullflexible,
	keywordstyle=\color{red}\bfseries,
	commentstyle=\color{blue},
	showstringspaces=false,
}

\newcommand{\Sum}{\mathit{sum}}


\title[ASD - Insiemi e dizionari]{\textbf{Algoritmi e Strutture Dati}\\[24pt]Insiemi e dizionari\\Riassunto finale}

\graphicspath{{figs/08/}}

\begin{document}

%-------------------------------------------------------------------------
\FrameTitle{}

%-------------------------------------------------------------------------
\begin{frame}{Insiemi e dizionari}

\vspace{-12pt}
\TwoCols{
\BB{Insiemi}
\BI
\item Collezione di oggetti
\EI
}{
\BB{Dizionari}	
\BI
\item Associazioni chiave-valore
\EI
}

\bigskip
\BB{Implementazione}
\BI 
\item Molte delle strutture dati viste finora
\item Vantaggi e svantaggi
\EI

\end{frame}




%-------------------------------------------------------------------------
\begin{frame}{Insiemi realizzati con vettori booleani}

\vspace{-12pt}
\TwoCols{
\BB{Insieme}
\BI
\item Interi $1 \ldots m$
\item Collezione di $m$ oggetti memorizzati in un vettore
\EI
}{
\BB{Rappresentazione}
\BI
\item Vettore booleano di $m$ elementi
\EI
}

\bigskip
\TwoCols{
\BB{Vantaggi}
\BI
\item Notevolmente semplice
\item Efficiente verificare se un elemento appartiene all’insieme
\EI
}{
\BB{Svantaggi}
\BI
\item Memoria occupata $O(m)$, indipendente dalle dimensioni effettive
\item Alcune operazioni inefficienti -- $O(m)$
\EI
}

\end{frame}

%-------------------------------------------------------------------------
\begin{frame}[shrink=12]{Insiemi realizzati con vettori booleani}

\vspace{-12pt}
\begin{Procedure}
\caption[A]{\Set (vettore booleano)}

\begin{multicols}{2}
$\BOOLEAN[\,]\ V$\;
$\INTEGER\ \setsize$\;
$\INTEGER\ \setcap$\;
\BlankLine

\alert{\Set}\ \PROCEDURE{\alert{\setconstructor(\INTEGER $m$)}}
{
  $\Set\ t = \NEW\ \Set$\;
  $t.\setsize = 0$\;
  $t.\setcap = m$\;
	$t.V = \NEW\ \INTEGER[1 \ldots m] = \{ \FALSE \}$\;
  \Return $t$\;
}
\BlankLine

\alert{\BOOLEAN}\ \PROCEDURE{\alert{\setcontains(\INTEGER $x$)}}
{
	\eIf{$1 \leq x \leq \setcap$}{
  	\Return\ $V[x]$\;
	}{
		\Return \FALSE\;
	}
}
\BlankLine

\alert{\INTEGER}\ \PROCEDURE{\alert{\textsf{size}()}}
{
  \Return\ $\setsize$\;
}
\BlankLine

\PROCEDURE{\alert{\setinsert(\INTEGER $x$)}}
{
	\If{$1 \leq x \leq \setcap$}{
  	\If{\NOT $V[x]$}{
    	$\setsize = \setsize+1$\;
  		$V[x] = \TRUE$\;
  	}
	}
}
\BlankLine

\PROCEDURE{\alert{\setremove(\INTEGER $x$)}}
{
	\If{$1 \leq x \leq \setcap$}{
  	\If{$V[x]$}{
    	$\setsize = \setsize-1$\;
			$V[x] = \FALSE$\;
  	}
  }
}
\end{multicols}
\BlankLine
\BlankLine
\end{Procedure}

\end{frame}

\begin{frame}[shrink=12]{Insiemi realizzati con vettori booleani}

\vspace{-12pt}
\begin{Procedure}
\caption[A]{\Set (vettore booleano)}

\alert{\Set}\ \PROCEDURE{\alert{\setunion(\Set $A$, \Set $B$)}}
{
  $\INTEGER\ \mathit{newsize} = \MAX(A.\setcap, B.\setcap)$\;
  $\Set\ C = \setconstructor(\mathit{newsize})$\;
  \For{$i = 1$ \TO\ $A.\setcap$}
  {
    \If{$A.\setcontains(i)$}{$C.\setinsert(i)$}
  }
  \For{$i = 1$ \TO\ $B.\setcap$}
  {
    \If{$B.\setcontains(i)$}{$C.\setinsert(i)$}
  }
  \Return $C$\;
}
\end{Procedure}

\end{frame}

\begin{frame}[shrink=12]{Insiemi realizzati con vettori booleani}

\vspace{-12pt}
\begin{Procedure}
\caption[A]{\Set (vettore booleano)}
\alert{\Set}\ \PROCEDURE{\alert{\setdifference(\Set $A$, \Set $B$)}}
{
  $\Set\ C = \setconstructor(A.\setcap)$\;
  \For{$i = 1$ \TO\ $A.\setcap$}
  {
    \If{$A.\setcontains(i)\ \AND\ \NOT\ B.\setcontains(i)$}{$C.\setinsert(i)$}
  }
  \Return $C$\;
}
\BlankLine
\alert{\Set}\ \PROCEDURE{\alert{\setintersection(\Set $A$, \Set $B$)}}
{
  $\INTEGER\ \mathit{newsize} = \MIN(A.\setcap, B.\setcap)$\;
  $\Set\ C = \setconstructor(\mathit{newsize})$\;
  \For{$i = 1$ \TO\ $\MIN.\setcap, B.\setcap)$}
  {
    \If{$A.\setcontains(i)\ \AND\ B.\setcontains(i)$}{$C.\setinsert(i)$}
  }
  \Return $C$\;
}
\end{Procedure}

\end{frame}

\begin{frame}{BitSet}

\BB{Java - \texttt{class java.util.BitSet}}

\medskip
{\footnotesize
\begin{tabular}{|l|l|}
\hline
\textbf{Metodo} & \textbf{Operaz.} \\\hline
\texttt{void and(BitSet set)} & Union \\\hline
\texttt{void or(BitSet set)} & Intersection \\\hline
\texttt{int cardinality()} & Set size \\\hline
\end{tabular}
\hfill
\begin{tabular}{|l|l|}
\hline
\textbf{Metodo} & \textbf{Operaz.} \\\hline
\texttt{void clear(int i)} & Remove \\\hline
\texttt{void set(int i)} & Insert \\\hline
\texttt{boolean get(int i)} & Contains \\\hline
\end{tabular}
}

\bigskip
\BB{C++ STL}

\BI
\item \alert{\texttt{std::bitset}} -- 
Stuttura dati bitset con dimensione fissata nel template al momento della compilazione.
\item \alert{\texttt{std::vector<bool>}} --
Specializzazione di \texttt{std::vector} per ottimizzare la memorizzazione, dimensione dinamica.
\EI


\end{frame}



\begin{frame}{Insiemi realizzati con liste / vettori non ordinati}
	
\begin{myboxtitle}[Costo operazioni]
\BI
\item Operazioni di ricerca, inserimento e cancellazione: \alert{$O(n)$}
\item Operazioni di inserimento (assumendo assenza): \alert{$O(1)$}
\item Operazioni di unione, intersezione e differenza: \alert{$O(nm)$}
\EI
\end{myboxtitle}


\begin{Procedure}
\caption[A]{\alert{\Set\ \setdifference($\Set\ A, \Set\ B$)}}

  $\Set\ C = \setconstructor()$\;
  \ForEach{$s \in A$}
  {
    \If{\NOT $B.\setcontains(s)$}
    {
       $C.\listinsert(s)$\;
    }
  }
  \Return\ $C$\;
\end{Procedure}
	

	
\end{frame}


\begin{frame}[shrink=5]{Insiemi realizzati con liste / vettori ordinati}
	
\vspace{-12pt}
\TwoColsCustom{0.62}{0.36}{
\vspace{-12pt}
\begin{Procedure}
\caption[A]{\alert{\List\ \setintersection(\List\ $A$, \List\ $B$)}}

  $\List\ C = \setconstructor()$\;
  $\Postype\ \Pos_a = A.\listhead()$\;
  $\Postype\ \Pos_b = B.\listhead()$\;
  \While{\NOT $A.\listend(\Pos_a)\ \AND\ \newline \makebox[1.15cm]{} \NOT\ B.\listend(\Pos_b)$}
  {
    \uIf{$A.\listread(\Pos_a) \Eq B.\listread(\Pos_b)$}
    {
       $C.\listinsert(C.\listtail(), A.\listread(\Pos_a))$\;
       $\Pos_a = A.\listsucc(\Pos_a)$\;
       $\Pos_b = B.\listsucc(\Pos_b)$\;
    }
    \uElseIf{$A.\listread(\Pos_a) < B.\listread(\Pos_b)$}
    {
       $\Pos_a = A.\listsucc(\Pos_a)$\;
    }
    \Else
    {
       $\Pos_b = B.\listsucc(\Pos_b)$\;
    }
  }
  \Return\ $C$\;
\end{Procedure}
}{
\BB{Costo operazioni}
\BIL
\item Ricerca: 
\BI
\item $O(n)$ (liste) 
\item $O(\log n)$ (vettori)
\EI
\item Inserimento e cancellazione
\BI
\item $O(n)$
\EI
\item Unione, intersezione e differenza: 
\BI
\item $O(n)$
\EI
\EIL
}
\end{frame}

%-------------------------------------------------------------------------
\begin{frame}{Insiemi -- Strutture dati complesse}
	
\vspace{-12pt}
\TwoColsCustom{0.46}{0.54}{
\begin{myboxtitle}[Alberi bilanciati]
\BIL
\item Ricerca, inserimento, cancellazione: \alert{$O(\log n)$}
\item Iterazione: \alert{$O(n)$}
\item Con ordinamento
\item Implementazioni:
\BI
\item Java \texttt{TreeSet}
\item Python \texttt{OrderedSet}
\item C++ STL \texttt{set}
\EI
\EIL
\end{myboxtitle}
}{
\begin{myboxtitle}[Hash table]
\BIL
\item Ricerca, inserimento, cancellazione: \alert{$O(1)$}
\item Iterazione: \alert{$O(m)$}
\item Senza ordinamento
\item Implementazioni:
\BI
\item Java \texttt{HashSet}
\item Python \texttt{set}
\item C++ STL \texttt{unordered\_set}
\EI
\EIL
\end{myboxtitle}
}	
	
\end{frame}

%-------------------------------------------------------------------------
\begin{frame}[shrink=20]{Insiemi -- Riassunto}
	
\renewcommand{\arraystretch}{1.3}

\begin{tabular}{|P{3.1cm}|P{1.5cm}|P{1.5cm}|P{1.5cm}|P{1.5cm}|P{1.7cm}|P{1.3cm}|}	
\hline
& \textsf{contains}\newline\textsf{lookup} & \textsf{insert} & \textsf{remove} & \textsf{min} & \textsf{foreach} & Ordine \\
\hline
Vettore booleano & $O(1)$ & $O(1)$ & $O(1)$ & $O(m)$ & $O(m)$ & Sì \\\hline
Lista non ordinata & $O(n)$ & $O(n)$ & $O(n)$ & $O(n)$ & $O(n)$ & No\\\hline
Lista ordinata & $O(n)$ & $O(n)$ & $O(n)$ & $O(1)$ & $O(n)$ & Sì\\\hline
Vettore ordinato & $O(\log n)$ & $O(n)$ & $O(n)$ & $O(1)$ & $O(n)$ & Sì\\\hline
Alberi bilanciati & $O(\log n)$ & $O(\log n)$ & $O(\log n)$ & $O(\log n)$ & $O(n)$ & Sì \\\hline
Hash\newline (Mem. interna) & $O(1)$ & $O(1)$ & $O(1)$ & $O(m)$ & $O(m)$ & No\\\hline
Hash\newline (Mem. esterna) & $O(1)$ & $O(1)$ & $O(1)$ & $O(m+n)$ & $O(m+n)$ & No \\\hline
\end{tabular}
	
\bigskip
$m \equiv$ dimensione del vettore o della tabella hash
	
\end{frame}



\begin{frame}{Python -- List}
\vspace{-12pt}
\begin{center}
\begin{tabular}{|P{2.7cm}P{2.0cm}|G{2.7cm}|G{2.7cm}|}
\hline
\multicolumn{2}{|c|}{\textbf{Operazione}}& \textbf{Caso medio} & \textbf{Caso pessimo ammortizzato} \\
\hline
\texttt{L.copy()} & \textcolor{blue}{Copy}  & $O(n)$ & $O(n)$ \\
\hline
\texttt{L.append(x)} & \textcolor{blue}{Append} & $O(1)$ & $O(1)$ \\
\hline
\texttt{L.insert(i,x)} & \textcolor{blue}{Insert} & $O(n)$ & $O(n)$ \\
\hline
\texttt{L.remove(x)} & \textcolor{blue}{Remove} & $O(n)$ & $O(n)$ \\
\hline
\texttt{L[i]} & \textcolor{blue}{Index} & $O(1)$ & $O(1)$ \\
\hline
\texttt{for x in L} & \textcolor{blue}{Iterator} & $O(n)$ & $O(n)$ \\
\hline
\texttt{L[i:i+k]} & \textcolor{blue}{Slicing} & $O(k)$ & $O(k)$ \\
\hline
\texttt{L.extend(s)} & \textcolor{blue}{Extend} &$O(k)$ & $O(k)$ \\
\hline
\texttt{x in L} & \textcolor{blue}{Contains} & $O(n)$ & $O(n)$ \\
\hline
\texttt{min(L)}, \texttt{max(L)} & \textcolor{blue}{Min, Max} & $O(n)$ & $O(n)$ \\
\hline
\texttt{len(L)} & \textcolor{blue}{Get length} &$O(1)$ & $O(1)$ \\
\hline
\end{tabular}

\bigskip
$n=\mathtt{len(L)}$
\end{center}
\end{frame}

\begin{frame}[shrink=3]{Python -- Set}
\vspace{-12pt}
\begin{center}
\begin{tabular}{|P{2.7cm}P{2.0cm}|G{2.7cm}|G{2.7cm}|}
\hline
\multicolumn{2}{|c|}{\textbf{Operazione}}& \textbf{Caso medio} & \textbf{Caso pessimo} \\
\hline
\texttt{x in S} & \textcolor{blue}{Contains} & $O(1)$ & $O(n)$ \\
\hline
\texttt{for x in S} & \textcolor{blue}{Iterator} & $O(n)$ & $O(n)$ \\
\hline
\texttt{S.add(x)} & \textcolor{blue}{Insert} & $O(1)$ & $O(n)$ \\
\hline
\texttt{S.remove(x)} & \textcolor{blue}{Remove} & $O(1)$ & $O(n)$ \\
\hline
\texttt{S|T} & \textcolor{blue}{Union} & $O(n + m)$ & $O(n \cdot m)$\\
\hline
\texttt{S\&T} & \textcolor{blue}{Intersection} & $O(\min(n, m))$ & $O(n \cdot m)$ \\
\hline
\texttt{S-T} & \textcolor{blue}{Difference} & $O(n)$ & $O(n \cdot m)$ \\
\hline
\end{tabular}

\bigskip
$n=\mathtt{len(S)}, m = \mathtt{len(T)}$
\end{center}
\end{frame}

\end{document}
%%%%%%%%%%%%%%%%%%%%%%%%%%%%%%%%%%%%%%%%%%%%%%%%%%%%%%%%%%%%%%%%%%%%%%%%%%


