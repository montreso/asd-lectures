\input templates/header

\usepackage{tikz}
\usepackage{xmpmulti}
\usepackage{listings}


\lstset{
  basicstyle=\ttfamily,
  columns=fullflexible,
	keywordstyle=\color{red}\bfseries,
	commentstyle=\color{blue},
	showstringspaces=false,
}

\newcommand{\Sum}{\mathit{sum}}


\title[ASD - Insiemi e dizionari]{\textbf{Algoritmi e Strutture Dati}\\[24pt]Insiemi e dizionari\\Riassunto finale}

\graphicspath{{figs/08/}}

\begin{document}

%-------------------------------------------------------------------------
\FrameTitle{}

%-------------------------------------------------------------------------
\begin{frame}{Insiemi e dizionari}

\vspace{-12pt}
\TwoCols{
\BB{Insiemi}
\BI
\item Collezione di oggetti
\EI
}{
\BB{Dizionari}	
\BI
\item Associazioni chiave-valore
\EI
}

\bigskip
\BB{Implementazione}
\BI 
\item Molte delle strutture dati viste finora
\item Vantaggi e svantaggi
\EI

\end{frame}




%-------------------------------------------------------------------------
\begin{frame}{Insiemi realizzati con vettori booleani}

\vspace{-12pt}
\TwoCols{
\BB{Insieme}
\BI
\item Interi $1 \ldots m$
\item Collezione di $m$ oggetti memorizzati in un vettore
\EI
}{
\BB{Rappresentazione}
\BI
\item Vettore booleano di $m$ elementi
\EI
}

\bigskip
\TwoCols{
\BB{Vantaggi}
\BI
\item Notevolmente semplice
\item Efficiente verificare se un elemento appartiene all’insieme
\EI
}{
\BB{Svantaggi}
\BI
\item Memoria occupata $O(m)$, indipendente dalle dimensioni effettive
\item Alcune operazioni inefficienti -- $O(m)$
\EI
}

\end{frame}

%-------------------------------------------------------------------------
\begin{frame}[shrink=12]{Insiemi realizzati con vettori booleani}

\vspace{-12pt}
\begin{Procedure}
\caption[A]{\Set (vettore booleano)}

\begin{multicols}{2}
$\BOOLEAN[\,]\ V$\;
$\INTEGER\ \setsize$\;
$\INTEGER\ \setcap$\;
\BlankLine

\PROCEDURE{\Set \setconstructor(\INTEGER $m$)}
{
  $\Set\ t = \NEW\ \Set$\;
  $t.\setsize = 0$\;
  $t.\setcap = m$\;
	$t.V = [ \FALSE] * m$\;
  \Return $t$\;
}
\BlankLine

\BOOLEAN\ \PROCEDURE{\setcontains(\INTEGER $x$)}
{
	\eIf{$1 \leq x \leq \setcap$}{
  	\Return\ $V[x]$\;
	}{
		\Return \FALSE\;
	}
}
\BlankLine

\INTEGER\ \PROCEDURE{size()}
{
  \Return\ $\setsize$\;
}
\BlankLine

\PROCEDURE{\setinsert(\INTEGER $x$)}
{
	\If{$1 \leq x \leq \setcap$}{
  	\If{\NOT $V[x]$}{
    	$\setsize = \setsize+1$\;
  		$V[x] = \TRUE$\;
  	}
	}
}
\BlankLine

\PROCEDURE{\setremove(\INTEGER $x$)}
{
	\If{$1 \leq x \leq \setcap$}{
  	\If{$V[x]$}{
    	$\setsize = \setsize-1$\;
			$V[x] = \FALSE$\;
  	}
  }
}
\end{multicols}
\BlankLine
\BlankLine
\end{Procedure}

\end{frame}

\begin{frame}[shrink=12]{Insiemi realizzati con vettori booleani}

\vspace{-12pt}
\begin{Procedure}
\caption[A]{\Set (vettore booleano)}

\bigskip
\TwoColsCustom{0.46}{0.54}{

%\begin{multicols}{2}
\Set\ \PROCEDURE{\setunion(\Set $A$, \Set $B$)}
{
  $\Set\ C = \setconstructor(\MAX(A.\setcap, B.\setcap))$\;
  \For{$i = 1$ \TO\ $A.\setcap$}
  {
    \If{$A.\setcontains(i)$}{$C.\setinsert(i)$}
  }
  \For{$i = 1$ \TO\ $B.\setcap$}
  {
    \If{$B.\setcontains(i)$}{$C.\setinsert(i)$}
  }
}
}{
\Set\ \PROCEDURE{\setdifference(\Set $A$, \Set $B$)}
{
  $\Set\ C = \setconstructor(A.\setcap)$\;
  \For{$i = 1$ \TO\ $A.\setcap$}
  {
    \If{$A.\setcontains(i)\ \AND\ \NOT\ B.\setcontains(i)$}{$C.\setinsert(i)$}
  }
}
}
\Set\ \PROCEDURE{\setintersection(\Set $A$, \Set $B$)}
{
  $\Set\ C = \setconstructor(\MIN(A.\setcap, B.\setcap))$\;
  \For{$i = 1$ \TO\ $\MIN.\setcap, B.\setcap)$}
  {
    \If{$A.\setcontains(i)\ \AND\ B.\setcontains(i)$}{$C.\setinsert(i)$}
  }
}
%\end{multicols}
\BlankLine
\BlankLine
\end{Procedure}

\end{frame}

\begin{frame}{BitSet}

\BB{Java - \texttt{class java.util.BitSet}}

\medskip
{\footnotesize
\begin{tabular}{|l|l|}
\hline
\textbf{Metodo} & \textbf{Operaz.} \\\hline
\texttt{void and(BitSet set)} & Union \\\hline
\texttt{void or(BitSet set)} & Intersection \\\hline
\texttt{int cardinality()} & Set size \\\hline
\end{tabular}
\hfill
\begin{tabular}{|l|l|}
\hline
\textbf{Metodo} & \textbf{Operaz.} \\\hline
\texttt{void clear(int i)} & Remove \\\hline
\texttt{void set(int i)} & Insert \\\hline
\texttt{boolean get(int i)} & Contains \\\hline
\end{tabular}
}

\bigskip
\BB{C++ STL}

\BI
\item \alert{\texttt{std::bitset}} -- 
Stuttura dati bitset con dimensione fissata nel template al momento della compilazione.
\item \alert{\texttt{std::vector<bool>}} --
Specializzazione di \texttt{std::vector} per ottimizzare la memorizzazione, dimensione dinamica.
\EI


\end{frame}



\begin{frame}{Insiemi realizzati con liste / vettori non ordinati}
	
\begin{myboxtitle}[Costo operazioni]
\BI
\item Operazioni di ricerca, inserimento e cancellazione: \alert{$O(n)$}
\item Operazioni di inserimento (assumendo assenza): \alert{$O(1)$}
\item Operazioni di unione, intersezione e differenza: \alert{$O(nm)$}
\EI
\end{myboxtitle}


\begin{Procedure}
\caption[A]{\Set\ \setdifference($\Set\ A, \Set\ B$)}

  $\Set\ C = \setconstructor()$\;
  \ForEach{$s \in A$}
  {
    \If{\NOT $B.\setcontains(s)$}
    {
       $C.\listinsert(s)$\;
    }
  }
  \Return\ $C$\;
\end{Procedure}
	

	
\end{frame}


\begin{frame}[shrink=5]{Insiemi realizzati con liste / vettori ordinati}

\vspace{-6pt}

\TwoColsCustom{0.52}{0.46}{
\vspace{-12pt}
\begin{Procedure}
\caption[A]{\List\ \setintersection(\List\ $A$, \List\ $B$)}

  $\List\ C = \setconstructor()$\;
  $\Postype\ p = A.\listhead()$\;
  $\Postype\ q = B.\listhead()$\;
  \While{\NOT $A.\listend(p)\ \AND\ \newline \makebox[1.15cm]{} \NOT\ B.\listend(q)$}
  {
    \uIf{$A.\listread(p) \Eq B.\listread(q)$}
    {
       $C.\listinsert(C.\listtail(), A.\listread(p))$\;
       $p = A.\listsucc(p)$\;
       $q = B.\listsucc(q)$\;
    }
    \uElseIf{$A.\listread(p) < B.\listread(q)$}
    {
       $p = A.\listsucc(p)$\;
    }
    \Else
    {
       $q = B.\listsucc(q)$\;
    }
  }
  \Return\ $C$\;
\end{Procedure}
}{
\BB{Costo operazioni}
\BIL
\item Ricerca: 
\BI
\item $O(n)$ (liste) 
\item $O(\log n)$ (vettori)
\EI
\item Inserimento/cancellazione
\BI
\item $O(n)$
\EI
\item Unione, intersezione e differenza: 
\BI
\item $O(n)$
\EI
\EIL
}
\end{frame}

%-------------------------------------------------------------------------
\begin{frame}{Insiemi -- Strutture dati complesse}
	
\TwoCols{
\begin{myboxtitle}[Alberi bilanciati]
\BIL
\item Ricerca, inserimento, cancellazione: \alert{$O(\log n)$}
\item Iterazione: \alert{$O(n)$}
\item Con ordinamento
\item Implementazioni:
\BI
\item Java \texttt{TreeSet}
\item Python \texttt{OrderedSet}
\item C++ STL \texttt{set}
\EI
\EIL
\end{myboxtitle}
}{
\begin{myboxtitle}[Hash table]
\BIL
\item Ricerca, inserimento, cancellazione: \alert{$O(1)$}
\item Iterazione: \alert{$O(m)$}
\item Senza ordinamento
\item Implementazioni:
\BI
\item Java \texttt{HashSet}
\item Python \texttt{set}
\item C++ STL \texttt{unordered\_set}
\EI
\EIL
\end{myboxtitle}
}	
	
\end{frame}

%-------------------------------------------------------------------------
\begin{frame}[shrink=20]{Insiemi -- Riassunto}
	
\renewcommand{\arraystretch}{1.3}

\begin{tabular}{|P{3.1cm}|P{1.5cm}|P{1.5cm}|P{1.5cm}|P{1.5cm}|P{1.7cm}|P{1.3cm}|}	
\hline
& \textsf{contains}\newline\textsf{lookup} & \textsf{insert} & \textsf{remove} & \textsf{min} & \textsf{foreach}\newline (Memoria) & Ordine \\
\hline
Vettore booleano & $O(1)$ & $O(1)$ & $O(1)$ & $O(m)$ & $O(m)$ & Sì \\\hline
Lista non ordinata & $O(n)$ & $O(n)$ & $O(n)$ & $O(n)$ & $O(n)$ & No\\\hline
Lista ordinata & $O(n)$ & $O(n)$ & $O(n)$ & $O(1)$ & $O(n)$ & Sì\\\hline
Vettore ordinato & $O(\log n)$ & $O(n)$ & $O(n)$ & $O(1)$ & $O(n)$ & Sì\\\hline
Alberi bilanciati & $O(\log n)$ & $O(\log n)$ & $O(\log n)$ & $O(\log n)$ & $O(n)$ & Sì \\\hline
Hash\newline (Mem. interna) & $O(1)$ & $O(1)$ & $O(1)$ & $O(m)$ & $O(m)$ & No\\\hline
Hash\newline (Mem. esterna) & $O(1)$ & $O(1)$ & $O(1)$ & $O(m+n)$ & $O(m+n)$ & No \\\hline
\end{tabular}
	
\bigskip
$m \equiv$ dimensione del vettore o della tabella hash
	
\end{frame}



\begin{frame}{Python -- List}
\vspace{-12pt}
\begin{center}
\begin{tabular}{|P{2.7cm}P{2.0cm}|G{2.7cm}|G{2.7cm}|}
\hline
\multicolumn{2}{|c|}{\textbf{Operazione}}& \textbf{Caso medio} & \textbf{Caso pessimo ammortizzato} \\
\hline
\texttt{L.copy()} & \textcolor{blue}{Copy}  & $O(n)$ & $O(n)$ \\
\hline
\texttt{L.append(x)} & \textcolor{blue}{Append} & $O(1)$ & $O(1)$ \\
\hline
\texttt{L.insert(i,x)} & \textcolor{blue}{Insert} & $O(n)$ & $O(n)$ \\
\hline
\texttt{L.remove(x)} & \textcolor{blue}{Remove} & $O(n)$ & $O(n)$ \\
\hline
\texttt{L[i]} & \textcolor{blue}{Index} & $O(1)$ & $O(1)$ \\
\hline
\texttt{for x in L} & \textcolor{blue}{Iterator} & $O(n)$ & $O(n)$ \\
\hline
\texttt{L[i:i+k]} & \textcolor{blue}{Slicing} & $O(k)$ & $O(k)$ \\
\hline
\texttt{L.extend(s)} & \textcolor{blue}{Extend} &$O(k)$ & $O(k)$ \\
\hline
\texttt{x in L} & \textcolor{blue}{Contains} & $O(n)$ & $O(n)$ \\
\hline
\texttt{min(L)}, \texttt{max(L)} & \textcolor{blue}{Min, Max} & $O(n)$ & $O(n)$ \\
\hline
\texttt{len(L)} & \textcolor{blue}{Get length} &$O(1)$ & $O(1)$ \\
\hline
\end{tabular}

\bigskip
$n=\mathtt{len(L)}$
\end{center}
\end{frame}

\begin{frame}[shrink=3]{Python -- Set}
\vspace{-12pt}
\begin{center}
\begin{tabular}{|P{2.7cm}P{2.0cm}|G{2.7cm}|G{2.7cm}|}
\hline
\multicolumn{2}{|c|}{\textbf{Operazione}}& \textbf{Caso medio} & \textbf{Caso pessimo} \\
\hline
\texttt{x in S} & \textcolor{blue}{Contains} & $O(1)$ & $O(n)$ \\
\hline
\texttt{S.add(x)} & \textcolor{blue}{Insert} & $O(1)$ & $O(n)$ \\
\hline
\texttt{S.remove(x)} & \textcolor{blue}{Remove} & $O(1)$ & $O(n)$ \\
\hline
\texttt{S|T} & \textcolor{blue}{Union} & $O(n + m)$ & $O(n \cdot m)$\\
\hline
\texttt{S\&T} & \textcolor{blue}{Intersection} & $O(\min(n, m))$ & $O(n \cdot m)$ \\
\hline
\texttt{S-T} & \textcolor{blue}{Difference} & $O(n)$ & $O(n \cdot m)$ \\
\hline
\end{tabular}

\bigskip
$n=\mathtt{len(S)}, m = \mathtt{len(T)}$
\end{center}
\end{frame}




\begin{frame}{BitSet + Tabelle Hash = Bloom Filters}

\vspace{-12pt}
\begin{columns}[c]
\begin{column}{0.48\textwidth}
\begin{myboxtitle}[BitSet]
\alert{Vantaggi}
\BI
\item 1 bit/oggetto
\EI
\medskip
\alert{Svantaggi}
\BI
\item Elenco prefissato di oggetti
\EI
\end{myboxtitle}

\end{column}
\hfill
\begin{column}{0.48\textwidth}
\begin{myboxtitle}[Tabelle Hash]
\alert{Vantaggi}
\BI
\item Struttura dati dinamica
\EI
\medskip
\alert{Svantaggi}
\BI
\item Alta occupazione di memoria
\EI
\end{myboxtitle}

\end{column}
\end{columns}

\medskip
\begin{myboxtitle}[Bloom filters]
\medskip
\TwoCols{
\alert{Vantaggi}
\BI
\item Struttura dati dinamica
\item Bassa occupazione di memoria
(10 bit/oggetto)
\EI
}
{
\alert{Svantaggi}
\BI
\item Niente cancellazioni
\item Risposta probabilistica
\item No memorizzazione
\EI
}
\end{myboxtitle}
\end{frame}

%-------------------------------------------------------------------------
\begin{frame}{Bloom filter}

\vspace{-6pt}
\begin{myboxtitle}[Specifica]
\BI
\item $\textsf{insert}(k)$: Inserisce l'elemento $x$ nel bloom filter
\item $\BOOLEAN\ \textsf{contains}(k)$\\
\BI
\item Se restituisce \FALSE, l'elemento $x$ è sicuramente non presente
nell'insieme
\item Se restituisce \TRUE, l'elemento $x$ può essere presente oppure no \\
(\alert{falsi positivi})
\EI
\EI
\end{myboxtitle}

\end{frame}

%-------------------------------------------------------------------------
\begin{frame}{Bloom filter}

\vspace{-6pt}
\BB{Trade-off fra occupazione di memoria e probabilità di falso positivo}
\TwoColsCustom{0.7}{0.26}{
\BI
\item Sia $\epsilon$ la probabilità di falso positivo
\item I bloom filter richiedono $1.44\log_2(1/\epsilon)$ bit per elemento
inserito
\EI
}{
\begin{tabular}{|c|r|}
\hline
$\epsilon$ & Bit \\
\hline
$10^{-1}$ & 4.78 \\
\hline
$10^{-2}$ & 9.57 \\
\hline
$10^{-3}$ & 14.35 \\
\hline
$10^{-4}$ & 19.13 \\
\hline
\end{tabular}
}
\end{frame}


\begin{frame}{Applicazioni dei Bloom Filter}
	
\BB{Chrome Safe Browsing}
\BIL
\item Chrome contiene un database delle URL associate a siti con malware, costantemente aggiornato
\item Fino al 2012, memorizzato con un Bloom Filter
\item Chrome verifica l'appartenenza di ogni URL al database
\BI
\item Se la risposta è \FALSE, non appartiene
\item Se la risposta è \TRUE, potrebbe appartenere e viene fatta una verifica
tramite un servizio centralizzato di Google
\EI
\EIL

\BB{Qualche dato (da prendere cum grano salis)}
\BI
\item Nel 2011, 650k URL memorizzati in 1.94MB
\item 25 bit per URL, $\epsilon \approx 10^{-5}$
\EI

\end{frame}

\begin{frame}{Applicazioni dei Bloom Filter}

\BB{Ogni qual volta una verifica locale permette di evitare un'operazione
più costosa, quali operazioni di I/O e comunicazioni di rete}

\begin{myboxtitle}[They say...]
\BI
\item \alert{Facebook} uses bloom filters for typeahead search, to fetch friends and friends of friends to a user typed query.
\item \alert{Apache HBase} uses bloom filter to boost read speed by filtering out unnecessary disk reads of HFile blocks which do not contain a particular row or column.
\item \alert{Transactional Memory} (TM) has recently applied Bloom filters to detect memory access conflicts among threads.
\item When you log into \alert{Yahoo} mail, the browser page requests a bloom filter representing your contact list
\EI
\end{myboxtitle}
	
\end{frame}

%-------------------------------------------------------------------------
\begin{frame}{Implementazione}

\vspace{-6pt}
\begin{overprint}
\onslide<1|handout:1>
\BI
\item Un vettore booleano $A$ di $m$ bit, inizializzato a \FALSE 
\item $k$ funzioni hash $h_1, h_2, \ldots, h_k: U \rightarrow [0, m-1]$
\EI
\onslide<2-4|handout:2>
\vspace{-12pt}
\begin{Procedure}
\caption[A]{$\textsf{insert}(k)$}
\For{$i = 1$ \TO\ $k$}{
  $A[h_i(k)] = \TRUE$\;
}
\end{Procedure}
\onslide<5|handout:3>
\vspace{-12pt}
\begin{Procedure}
\caption[A]{$\BOOLEAN\ \textsf{contains}(k)$}
\For{$i = 1$ \TO\ $k$}{
  \lIf{$A[h_i(k)] \Eq \FALSE$}{
		\Return \FALSE
	}
}
\Return \TRUE\;
\end{Procedure}
\end{overprint}

\vspace{-6pt}
\begin{overprint}
\includegraphics<1|handout:1>[width=1.0\textwidth,page=1]{bloom-crop.pdf}
\includegraphics<2|handout:0>[width=1.0\textwidth,page=2]{bloom-crop.pdf}
\includegraphics<3|handout:0>[width=1.0\textwidth,page=3]{bloom-crop.pdf}
\includegraphics<4|handout:2>[width=1.0\textwidth,page=4]{bloom-crop.pdf}
\includegraphics<5|handout:3>[width=1.0\textwidth,page=5]{bloom-crop.pdf}
\end{overprint}

\end{frame}

\begin{frame}{Qualche formula (senza dimostrazione)}
\BI
\item Dati $n$ oggetti, $m$ bit, $k$ funzioni hash, la probabilità di un falso
positivo è pari a:

\[
 \epsilon = \left( 1 - e^{-kn/m} \right)^k
\]
\item Dati $n$ oggetti e $m$ bit, il valore ottimale per $k$ è pari a

\[
  k = \frac{m}{n}\ln 2
\]
\item Dati $n$ oggetti e una probabilità di falsi positivi $\epsilon$, il numero
di bit $m$ richiesti è pari a:

\[
 m = -\frac{n \ln \epsilon}{(\ln 2)^2}
\]
\EI

\end{frame}


\end{document}
%%%%%%%%%%%%%%%%%%%%%%%%%%%%%%%%%%%%%%%%%%%%%%%%%%%%%%%%%%%%%%%%%%%%%%%%%%


