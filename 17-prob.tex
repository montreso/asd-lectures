\input templates/header
\title[ASD - Algoritmi probabilistici]{\textbf{Algoritmi e Strutture Dati}\\[24pt]Algoritmi probabilistici}

\usepackage[mode=buildnew]{standalone}
\usepackage{xcolor}
\usepackage{colortbl}
\usepackage{epigraph}
\usepackage{tikz}
\usepackage{xmpmulti}
\usepackage{listings}

\lstset{
  basicstyle=\ttfamily,
  columns=fullflexible,
  keywordstyle=\color{red}\bfseries,
  commentstyle=\color{blue},
  showstringspaces=false,
}

\newcommand{\R}[1]{\textcolor{red}{#1}}
\newcommand{\B}[1]{\textcolor{blue}{#1}}

\renewcommand{\Primo}{\mathit{start}\xspace}
\renewcommand{\Ultimo}{\mathit{end}\xspace}

\renewcommand{\arraystretch}{1.4}
\graphicspath{{figs/17/}}


\begin{document}


%-------------------------------------------------------------------------
\FrameTitle{}

%-------------------------------------------------------------------------
\FrameContent



%%%%%%%%%%%%%%%%%%%%%%%%%%%%%%%%%%%%%%%%%%%%%%%%%%%%%%%%%%%%%%%%%%%%%%%%%%
\section{Introduzione}

%-------------------------------------------------------------------------
\begin{frame}{Introduzione}

\BB{“Audentes furtuna iuvat” - Virgilio}
\BIL
\item Se non sapete quale strada prendere, fate una scelta casuale
\EIL

\BB{Casualità negli algoritmi visti finora}
\BIL
\item Analisi del caso medio
  \BI
  \item Si calcola la media su tutti i possibili dati di ingresso
  in base ad una distribuzione di probabilità
  \item Esempio: caso medio Quicksort, si assume che tutte 
  le permutazioni siano equiprobabili
  \EI
\item Hashing
  \BI 
  \item Le funzion hash equivalgono a "randomizzare" le chiavi
  \item Distribuzione uniforme
  \EI
\EIL

\end{frame}

%-------------------------------------------------------------------------
\begin{frame}{Introduzione}

\BB{Algoritmi probabilistici}
Il calcolo delle probabilità è applicato non ai dati di input, ma ai dati di output
\BIL
\item  Algoritmi corretti, il cui tempo di funzionamento è probabilistico (\alert{Las Vegas})
\item Algoritmi la cui correttezza è probabilistica (\alert{Montecarlo})
\EIL

\end{frame}

\section{Algoritmi Montecarlo}

%-------------------------------------------------------------------------
\begin{frame}{Test di primalità}

\vspace{-9pt}
\begin{block}{Piccolo teorema di Fermat}
Per ogni numero primo $n$ e ogni $b \in [2, \ldots, n-1]$:
$
b^{n-1} \bmod n = 1
$
\end{block}

\begin{columns}[T]
\column{0.45\textwidth}
\begin{overprint}
\onslide<1-2|handout:1>
\BB{Test di primalità di Fermat (V1)}
\begin{Procedure}
\caption[A]{\textsf{isPrime}(\INTEGER\ $n$)}
  $b = \textsf{random}(2, n-1)$\;
  \If{$b^{n-1} \bmod n \neq 1$}{
    \Return \FALSE\;
  }{
    \Return \TRUE\;
  }
\end{Procedure}
\onslide<3-4|handout:2>
\BB{Test di primalità di Fermat (V2)}
\begin{Procedure}
\caption[A]{\textsf{isPrime}(\INTEGER\ $n$)}
\For{$i=1$ \TO\ $k$}{
  $b = \textsf{random}(2, n-1)$\;
  \If{$b^{n-1} \bmod n \neq 1$}{
    \Return \FALSE\;
  }
}
\Return \TRUE\;
\end{Procedure}
\end{overprint}
\column{0.48\textwidth}
\onslide<1-4|handout:1-2>
\BB{Questo algoritmo è corretto?}
\begin{overprint}
\onslide<2|handout:1>
Esistono numeri composti tali che:

\[
\alert{\exists} b \in [2, \ldots, n-1]: b^{n-1} \bmod n = 1
\]

\begin{tabular}{|l|l|}
\hline
Output & $n$ \\\hline    
\FALSE & \alert{sicuramente} composto \\\hline
\TRUE & \alert{possibilmente} primo \\\hline
\end{tabular}

\onslide<4|handout:2>
\medskip
\begin{tabular}{|l|l|}
\hline
Output & $n$ \\\hline    
\FALSE & \alert{sicuramente} composto \\\hline
\TRUE & \alert{probabilmente} primo \\\hline
\end{tabular}

\medskip
Esistono dei numeri composti (\alert{numeri di Carmichael}),
tali che:

\[
\alert{\forall} b \in [2, \ldots, n-1]: b^{n-1} \bmod n = 1
\]
\end{overprint}
\column{0.02\textwidth}
\end{columns}

\end{frame}

%-------------------------------------------------------------------------
\begin{frame}{Test di Miller-Rabin}

\BIL
\item Esprimiamo $n-1$ come: $n-1 = m 2^v$ con $m$ dispari
  \BI
  \item la rappresentazione binaria di $n-1$ è uguale alla rappresentazione 
    binaria del numero dispari $m$ seguito da $v$ zeri
  \EI
\item Sia $b$ un numero in $[1, \ldots, n-1]$
\EIL

\bigskip
\begin{overprint}
\onslide<1|handout:1>
\BB{Se un numero $n$ è \alert{primo}, allora valgono entrambe le seguenti condizioni}
\onslide<2|handout:2>
\BB{Se un numero $n$ è \alert{composto}, allora almeno una delle seguenti condizioni è falsa}
\end{overprint}
\BEL
\item $\textsf{mcd}(n,b) = 1$
\item $b^m \bmod n = 1$ oppure $\exists i, 0 \leq i \leq v-1: b^{2^im} \bmod n = -1$
\EEL
      
\end{frame}

%-------------------------------------------------------------------------
\begin{frame}{Test di Miller-Rabin}

\BIL
\item Rabin ha dimostrato che se $n$ è composto, allora ci sono
almeno $3/4(n-1)$ valori in $[1, \ldots, n-1]$ per i quali una delle condizioni sopra è falsa
\item Il test di compostezza ha una probabilità inferiore a $1/4$ di rispondere
erroneamente
\EIL

\begin{Procedure}
\caption[A]{\textsf{isPrime}(\INTEGER\ $n$)}
\For{$i=1$ \TO\ $k$}{
  $b = \textsf{random}(2, n-1)$\;
  \If{$\textsf{isComposite}(n,b)$}{
    \Return \FALSE\;
  }
}
\Return \TRUE\;
\end{Procedure}

\end{frame}

%-------------------------------------------------------------------------
\begin{frame}{Test di Miller-Rabin}

\BB{Riassunto}
\BIL
\item Complessità: $O(k \log^2 n \log \log n \log \log \log n)$
\item Probabilità di errore: $(1/4)^k$
\item Algoritmo di tipo Montecarlo
\EIL

\bigskip
\BB{Algoritmi probabilistici vs algoritmi deterministici}
\BIL
\item Dal 2002, esiste l'algoritmo deterministico AKS di complessità $O(\log^{6+\epsilon})$
\item I fattori moltiplicativi coinvolti sono molto alti
\item Si preferisce quindi l'algoritmo di Miller-Rabin
\EIL

\end{frame}

%-------------------------------------------------------------------------
\begin{frame}{Esempio -- Espressione polinomiale nulla}

\vspace{-9pt}
\begin{block}{Problema}
Data un’espressione algebrica polinomiale $p(x_1 , \ldots , x_n)$ 
in $n$ variabili, determinare se $p$ è identicamente nulla oppure no.
\end{block}

\bigskip
Discussione
\BIL
\item Assumiamo che non sia in forma di monomi - altrimenti è banale
\item Gli algoritmi basati su semplificazioni sono molto complessi
\EIL

\end{frame}


%-------------------------------------------------------------------------
\begin{frame}{Esempio -- Espressione polinomiale nulla}

\BB{Algoritmo}
\BIL
\item Si genera una $n$-pla di valori $v_1, \ldots, v_n$
\item Si calcola $x= p(v_1 , \ldots , v_n)$
  \BI
	\item Se $x \neq 0$, $p$ non è identicamente nulla
	\item Se $x = 0$, $p$ \alert{potrebbe} essere identicamente nulla
	\EI
\item Se ogni $v_i$ è un valore intero compreso casuale fra $1$ e $2d$, dove $d$ è 
il grado del polinomio, allora la probabilità di errore non supera $1/2$.
\item Si ripete $k$ volte, riducendo la probabilità di errore a $(1/2)^k$
\EIL

\end{frame}

\section{Algoritmi Las Vegas}

%-------------------------------------------------------------------------
\begin{frame}{Statistica}

\begin{block}{Algoritmi statistici su vettori}
Estraggono alcune caratteristiche statisticamente rilevanti da un vettore numerico 
\end{block}

\medskip
\BB{Esempi}
\BIL
\item \alert{Media}: $\mu = \frac{1}{n} 	\sum_{i=1}^n A[i]$
\item \alert{Varianza}: $\sigma^2 = \frac{1}{n} \sum_{i=1}^n (A[i]- \mu)^2$
\item \alert{Moda}: il valore (o i valori) più frequenti
\EIL
\end{frame}

%-------------------------------------------------------------------------
\begin{frame}{Statistiche d'ordine}

\begin{block}{Selezione}
Dato un array $A$ contenente $n$ valori e un valore $1 \leq k \leq n$, 
trovare l'elemento che si occuperebbe la posizione $k$ se il vettore fosse ordinato
\end{block}

\medskip
\begin{block}{Mediana}
Il problema del calcolo della mediana è un sottoproblema del problema
della selezione con $k=\lceil n/2 \rceil$.
\end{block}

\end{frame}

%-------------------------------------------------------------------------
\begin{frame}{Selezione per piccoli valori di $k$}

\BB{Intuizione}
\BIL
\item L'albero del torneo può essere “simulato” da uno heap
\item L'algoritmo può essere generalizzato a valori generici di $k > 2$
\EIL

\TwoColsCustom{0.55}{0.42}{
\begin{Procedure}
\caption[A]{\textsf{heapSelect}(\Item[\,]\ $A$, \INTEGER $n$, \INTEGER $k$)}
  $\textsf{buildHeap}(A)$\;
  \For{i = 1 \TO\ k-1}{
    $\textsf{deleteMin}(A, n)$\;
  }
  \Return $\textsf{deleteMin}(A,n)$\;
\end{Procedure}
}{
\medskip
\BB{Complessità computazionale}
\BIL
\item $O(n + k \log n)$
\item Se $k = O(n/\log n)$,\\ il costo è $O(n)$
\item Non va bene per $k = n/2$
\EIL
}
\end{frame}
	
%-------------------------------------------------------------------------
\begin{frame}{}

\BB{Idea}
\BIL
\item Approccio divide-et-impera simile al Quicksort
\item Essendo un problema di ricerca, non è necessario cercare in entrambe le partizioni, basta cercare in una sola di esse
\item Bisogna fare attenzione agli indici
\EIL

\end{frame}

%-------------------------------------------------------------------------
\begin{frame}{Algoritmo di selezione}

\vspace{-12pt}
\small
\begin{Procedure}
\caption[A]{\Item \textsf{selection}($\Item[\,]\ A$, \INTEGER $\Primo$, \INTEGER $\Ultimo$, \INTEGER $k$)}
\eIf{$\Primo == \Ultimo$}{
  \Return $A[\Primo]$\;
}{
  $\INTEGER\ j = \textsf{pivot}(A, \Primo, \Ultimo)$\;
  $\INTEGER\ q = j - \Primo + 1$\;
  \uIf{$k==q$}{
    \Return $A[j]$\;
  }
  \uElseIf{$k<q$}{
    \Return $\textsf{selection}(A, \Primo, j-1, k)$\;
  }
  \Else{
    \Return $\textsf{selection}(A, j+1, \Ultimo, k-q)$\;
  }
}
\end{Procedure}
\vspace{-18pt}
\IG{0.8}{mediana.pdf}

\end{frame}

%-------------------------------------------------------------------------
\begin{frame}{Complessità}

\vspace{-9pt}
\TwoCols{
\BB{Caso pessimo}
\begin{align*}
  T(n) &= \begin{cases}
    1 & n \leq 1 \\
    T(n-1) + n & n>1 \\
  \end{cases} \\
  T(n) &= O(n^2)
\end{align*}  
}{
\BB{Caso ottimo}
\begin{align*}
  T(n) &= \begin{cases}
    1 & n \leq 1 \\
    T(n/2) + n & n>1 \\
  \end{cases} \\
  T(n) &= O(n)
\end{align*}  
}

\bigskip
\BB{Caso medio}
Assumiamo che \textsf{pivot}() restituisca con la stessa probabilità una qualsiasi posizione $j$ del vettore $A$
\begin{alignat*}{2}
T(n) &= n-1 + \frac{1}{n}\sum_{q=1}^{n} T\left(\max\{q-1, n-q\}\right)\\
     &\le n-1 + \frac{2}{n} \sum_{q=\lfloor n/2\rfloor}^{n-1} T(q), &&\qquad\textrm{per $n > 1$}
\end{alignat*}


\end{frame}

%-------------------------------------------------------------------------
\begin{frame}{Complessità}

\begin{align*}
T\left(n\right) &\le n - 1 + \frac{2c}{n} \sum_{q=\left\lfloor n/2\right\rfloor}^{n-1}q \\
&\le n + \frac{2c}{n}\left(\sum_{q=1}^{n-1}q - \sum_{q=1}^{\left\lfloor n/2\right\rfloor-1}q\right)\\
&= n + \frac{2c}{n}\left(\frac{n\left(n- 1\right)}{2} - \frac{\left\lfloor n/2\right\rfloor\left(\left\lfloor n/2\right\rfloor - 1\right)}{2}\right)\\
&\le n + \frac{c}{n}\big(n\left(n- 1\right) - \left(n/2+1\right)\left(n/2\right)\big)\\
&= n + c\left(n - 1\right) - \left(c/2\right)\left(n/2+ 1\right) = n + cn - c - cn/4 - c/2\\
&= cn \left( \frac{1}{c}+ \frac{3}{4} - \frac{3}{2n}\right) \leq cn \left( \frac{1}{c}+ \frac{3}{4}\right) \leq cn
\end{align*}


\end{frame}

%-------------------------------------------------------------------------
\begin{frame}{Complessità}
   
\BIL
\item  Siamo partiti dall’assunzione 
  \BI
  \item $j$ assume equiprobabilisticamente tutti i valori compresi fra $1$ e $n$
  \EI
\item E se non fosse vero?
\item Lo forziamo noi!
  \BI
  \item  $A[\textsf{random}(\Primo, \Ultimo)] \leftrightarrow A[\Primo]$
  \EI
\item Questo accorgimento vale anche per QuickSort    
\item La complessità nel caso medio:
  \BI
  \item $O(n)$ nel caso della Selezione
  \item $O(n \log n)$ nel caso dell'Ordinamento
  \EI
\EIL

\end{frame}

\section{}

%-------------------------------------------------------------------------
\begin{frame}{Selezione deterministica}

\BB{Algoritmo black-box}
Supponiamo di avere un algoritmo “black box” che mi ritorni il mediano di $n$ valori in tempo $O(n)$

\bigskip
\BB{Domande}
\BIL
\item Potrei utilizzarlo per ottimizzare il problema della selezione?
\item Che complessità otterrei?
\EIL

\end{frame}

%-------------------------------------------------------------------------
\begin{frame}{Selezione deterministica}

\BB{Se conoscessi tale algoritmo}
\BIL
\item il problema della selezione sarebbe quindi risolto...
\item ... ma dove lo trovo un simile algoritmo?
\EIL

\bigskip
\BB{Rilassiamo le nostre pretese}
\BIL
\item Supponiamo di avere un algoritmo “black box” che mi ritorni un valore che dista al più $\frac{3}{10}n$ dal mediano (nell'ordinamento)
\item Potrei utilizzarlo per ottimizzare il problema della selezione?
\item Che complessità otterrei?
\EIL

\end{frame}

\begin{frame}{Selezione deterministica}

\BB{Idea}
\BIL
\item Suddividi i valori in gruppi di 5. Chiameremo l'$i$-esimo gruppo $S_i$, con $i \in \left[1, \lceil n/5 \rceil \right]$
\item Trova il mediano $M_i$ di ogni gruppo $S_i$
\item Tramite una chiamata ricorsiva, trova il mediano $m$ dei mediani
$[M_1, M_2, \ldots, M_{\lceil n/5 \rceil}]$
\item Usa $M$ come pivot e richiama l'algoritmo ricorsivamente sull'array opportuno, come nella \textsf{selection}() randomizzata
\item Quando la dimensione scende sotto una certa dimensione, possiamo
utilizzare un algoritmo di ordinamento per trovare il mediano
\EIL

\end{frame}

\begin{frame}{Selezione deterministica}

\vspace{-12pt}
\small
\begin{Procedure}
\caption[A]{\Item\ \textsf{select}($\Item[\,]\ A$, \INTEGER $\Primo$, \INTEGER $\Ultimo$, \INTEGER $k$)}
\% Se la dimensione è inferiore ad una soglia (10), ordina il vettore e\;
\% restituisci il $k$-esimo  elemento di $A[\Primo \ldots \Ultimo]$\;
\If{$\Ultimo-\Primo+1 \leq 10$}{
  $\textsf{InsertionSort}(A, \Primo, \Ultimo)$\Comment*[f]{Versione con indici inizio/fine}\;
  \Return $A[\Primo+k-1]$\;
}
\BlankLine
\% Divide $A$ in $\lceil n/5 \rceil$ sottovettori di dim. $5$ e ne calcola la mediana\;
$M = \NEW\ \INTEGER[1 \ldots \lceil n/5 \rceil]$\;
\For{$i = 1$ \TO $\lceil n/5 \rceil$}{
  $M[i] = \textsf{median5}(A, \Primo+(i-1) \cdot 5, \Ultimo)$\;
}
\BlankLine
\% Individua la mediana delle mediane e usala come perno\;
$\Item\ m = \textsf{select}(M, 1, \lceil n/5 \rceil, \lceil \lceil n/5 \rceil/2 \rceil)$\;
$\INTEGER\ j = \textsf{pivot}(A, \Primo, \Ultimo, m)$\Comment*[f]{Versione con $m$ in input}\;
[...]\;
\end{Procedure}

\end{frame}

\begin{frame}{Selezione deterministica}

\vspace{-12pt}
\small
\begin{Procedure}
\caption[A]{\Item\ \textsf{select}($\Item[\,]\ A$, \INTEGER $\Primo$, \INTEGER $\Ultimo$, \INTEGER $k$)}
[...]\;

\BlankLine
\% Calcola l'indice $q$ di $m$ in $[\Primo \ldots \Ultimo]$\;
\% Confronta $q$ con l'indice cercato e ritorna il valore conseguente\;
$\INTEGER\ q = j-\Primo+1$\;
\uIf{$q==k$}{
  \Return $m$\;
}
\uElseIf{$q < k$}{
  \Return $\textsf{select}(A, \Primo, q-1, k)$\;
}
\Else{
  \Return $\textsf{select}(A, q+1, \Ultimo, k-q)$\;
}
\end{Procedure}

\end{frame}


\begin{frame}{Selezione deterministica}

\BIL
\item  Il calcolo dei mediani $M[]$ richiede al più $6 \lceil n/5 \rceil$  confronti.
\item La prima chiamata ricorsiva dell'algoritmo $\textsf{select}()$ viene effettuata su $\lceil n/5 \rceil$ elementi
\item La seconda chiamata ricorsiva dell'algoritmo $\textsf{select}()$ viene effettuata al massimo  su $7n/10$ elementi\\
 (esattamente $n - 3 \lceil  \lceil n/5 \rceil /2 \rceil$) 
\item L'algoritmo $\textsf{select}()$ esegue nel caso pessimo $O(n)$ confronti

\[
  T(n) = T(n/5) + T(7n/10) + 11/5n
\]
\EIL

\begin{center}
\begin{tabular}{|c|c|c|c|c||c|c|c|c|c||c|c|c|c|c|}
\hline
\alert{1} & \alert{4} & \alert{\textbf{7}} & 10 & 13 & \alert{2} & \alert{5} & \alert{\textbf{8}} & 11 & 14 & 3 & 6 & \textbf{9} & 12 & 15 \\\hline
\end{tabular}
\end{center}

\end{frame}

\begin{frame}{Conclusioni}

\BB{Quale preferire?}
\BIL
\item Algoritmo probabilistico Las Vegas in tempo atteso $O(n)$
\item Algoritmo deterministico in tempo $O(n)$, con fattori moltiplicativi
  più alti
\EIL


\bigskip
\BB{Note storiche}
\BIL
\item Nel 1883 Lewis Carroll (!) notò che il secondo premio nei tornei di tennis non veniva assegnato in maniera equa.
\item Nel 1932, Schreier dimostrò che $n + \log n - 2$  incontri sono sempre sufficienti per trovare il secondo posto
\item Nel 1973, a opera di Blum, Floyd, Pratt, Rivest e Tarjan, appare il primo algoritmo deterministico
\EIL
\end{frame}

\end{document}





