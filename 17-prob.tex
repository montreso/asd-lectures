\input templates/header
\title[ASD - Algoritmi probabilistici]{\textbf{Algoritmi e Strutture Dati}\\[24pt]Algoritmi probabilistici}

\usepackage[mode=buildnew]{standalone}
\usepackage{xcolor}
\usepackage{colortbl}
\usepackage{epigraph}
\usepackage{tikz}
\usepackage{xmpmulti}
\usepackage{listings}
\usepackage{xfrac}
\usepackage{cancel}

\lstset{
  basicstyle=\ttfamily,
  columns=fullflexible,
  keywordstyle=\color{red}\bfseries,
  commentstyle=\color{blue},
  showstringspaces=false,
}

\newcommand{\R}[1]{\textcolor{red}{#1}}
\newcommand{\B}[1]{\textcolor{blue}{#1}}

\renewcommand{\arraystretch}{1.4}
\graphicspath{{figs/17/}}


\begin{document}


%-------------------------------------------------------------------------
\FrameTitle{}

%-------------------------------------------------------------------------
\FrameContent



%%%%%%%%%%%%%%%%%%%%%%%%%%%%%%%%%%%%%%%%%%%%%%%%%%%%%%%%%%%%%%%%%%%%%%%%%%
\section{Introduzione}

%-------------------------------------------------------------------------
\begin{frame}{Introduzione}

\vspace{-9pt}
\BB{“Audentes furtuna iuvat” - Virgilio}
\BIL
\item Se non sapete quale strada prendere, fate una scelta casuale
\EIL

\BB{Casualità negli algoritmi visti finora}
\BIL
\item Analisi del caso medio
  \BI
  \item Si calcola la media su tutti i possibili dati di ingresso
  in base ad una distribuzione di probabilità
  \item Esempio: caso medio Quicksort, si assume che tutte 
  le permutazioni siano equiprobabili
  \EI
\item Hashing
  \BI 
  \item Le funzioni hash equivalgono a "randomizzare" le chiavi
  \item Distribuzione uniforme
  \EI
\EIL

\end{frame}

%-------------------------------------------------------------------------
\begin{frame}{Introduzione}

\vspace{-9pt}
\begin{myboxtitle}[Algoritmi probabilistici]
Il calcolo delle probabilità è applicato non ai dati di input, ma ai dati di output
\BIL
\item Algoritmi la cui correttezza è probabilistica (\alert{Montecarlo})
\item  Algoritmi corretti, il cui tempo di funzionamento è probabilistico (\alert{Las Vegas})
\EIL
\end{myboxtitle}

\end{frame}

\section{Algoritmi Montecarlo}

\subsection{Test di primalità}


\begin{frame}{Test di primalità (Primality test)}
    
\vspace{-9pt}
\begin{myboxtitle}[Test di primalità]
Algoritmo per determinare se un numero in input $n$ è primo.
\end{myboxtitle}

\begin{myboxtitle}[Fattorizzazione]
Algoritmo per listare i fattori che compongono un numero composto.
\end{myboxtitle}

\vspace{-6pt}
\TwoCols{

\begin{Procedure}
\caption{\BOOLEAN \textsf{isPrimeNaif}(\INTEGER $n$)}
\For{$i = 2$ \TO $\lfloor \sqrt{n} \rfloor$}{
  \If{$n/i \Eq \lfloor n/i \rfloor$}{
    \Return \FALSE\;
  }
}
\Return \TRUE\;
\end{Procedure}
}{
\bigskip
Una soluzione inefficiente testa tutti gli interi fra 2 e $\lfloor \sqrt{n} \rfloor$
}
\end{frame}


%-------------------------------------------------------------------------
\begin{frame}{Test di primalità -- Fermat}

\vspace{-9pt}
\begin{myboxtitle}[Piccolo teorema di Fermat]
Se $n$ è primo, allora:
\smallskip
\[
\forall b, 2 \leq b < n: \qquad  b^{n-1} \bmod n = 1 
\]
\end{myboxtitle}

\begin{columns}[T]
\column{0.45\textwidth}
\begin{overprint}
\onslide<1-3|handout:1>
\BB{Test di primalità di Fermat}
\begin{Procedure}
\caption[A]{\BOOLEAN\ \textsf{isPrime1}(\INTEGER\ $n$)}
  $b = \textsf{random}(2, n-1)$\;
  \If{$b^{n-1} \bmod n \neq 1$}{
    \Return \FALSE\;
  }{
    \Return \TRUE\;
  }
\end{Procedure}
\onslide<4-5|handout:2>
\BB{Test di primalità di Fermat}
\begin{Procedure}
\caption[A]{\BOOLEAN\ \textsf{isPrime2}(\INTEGER\ $n$)}
\For{$i=1$ \TO\ $k$}{
  $b = \textsf{random}(2, n-1)$\;
  \If{$b^{n-1} \bmod n \neq 1$}{
    \Return \FALSE\;
  }
}
\Return \TRUE\;
\end{Procedure}
\end{overprint}
\column{0.51\textwidth}
\onslide<1-5|handout:1-2>
\vspace{-8pt}
\BB{Questo algoritmo è corretto?}
\begin{overprint}
\onslide<2|handout:1>


% \medskip
Esistono numeri composti (\alert{pseudoprimi} in base $b$) tali che:
\smallskip
\[
\alert{\exists} b, 2 \leq b < n-1: b^{n-1} \bmod n = 1
\]

Esempio :
\smallskip
\begin{align*}
& n = 341 = 11 \times 13 \\
&  2^{340} \bmod 341 = 1 
\end{align*}

\onslide<3|handout:2>

\smallskip
\begin{tabular}{|l|l|}
\hline
Output & $n$ \\\hline
\FALSE & \alert{sicuramente} composto \\\hline
\TRUE & \alert{possibilmente} primo \\\hline
\end{tabular}


\onslide<5|handout:3>

\smallskip
Esistono numeri composti tali che:
\smallskip
\[
\alert{\forall} b, 2 \leq b < n-1: b^{n-1} \bmod n = 1
\]
(\alert{Numeri di Carmichael})
\end{overprint}
\column{0.02\textwidth}
\end{columns}

\end{frame}

%-------------------------------------------------------------------------
\begin{frame}{Test di primalità -- Miller-Rabin}

\vspace{-9pt}
\begin{myboxtitle}[Teorema]
Se $n$ è \textcolor{blue}{primo} , allora \alert{per ogni} intero $b$, con $2 \leq b < n$ valgono \alert{entrambe} le seguenti condizioni:
\BEL
\item $\textrm{mcd}(n,b) \alert{=} 1$
\item $b^m \bmod n \alert{=} 1 \alert{\vee \exists} i, 0 \leq i < v: b^{m \cdot 2^i} \bmod n \alert{=} n-1$
\EEL
con $n-1=m \cdot 2^v$, $m$ dispari
\end{myboxtitle}

\begin{myboxtitle}[Contrapposizione]
Se \alert{esiste} un intero $b$, con $2 \leq b < n$ tale che \alert{almeno una} delle seguenti condizioni è vera:
\BEL
\item $\textrm{mcd}(n,b) \alert{\neq} 1$
\item $b^m \bmod n \alert{\neq} 1 \alert{\wedge \forall} i, 0 \leq i < v: b^{m \cdot 2^i} \bmod n \alert{\neq} n-1$
\EEL
allora $n$ è \textcolor{blue}{composto}
\end{myboxtitle}




\end{frame}

%-------------------------------------------------------------------------
\begin{frame}{Test di primalità -- Miller-Rabin}


\vspace{-9pt}
\begin{myboxtitle}[Verifica primalità $n$ -- Esempio: $n=221 = 13 \times 17$]
\small
\begin{tabular}{m{6cm}m{5cm}}
Sia $n$ un numero dispari & $n=221$ \\
%
Sia $n-1 = m \cdot 2^v$, con $m$ dispari & $220 = 55 \cdot 2^2$ \\
%
$n-1$ in binario è pari a:\newline 
$m$ in binario seguito da $v$ zeri & $\overbrace{110111}^{m=55}\underbrace{00}_{v=2}$\\
%
se $\exists b, 2 \leq b < n$ per cui una delle seguenti affermazioni è vera, $n$ è \alert{composto}: & $b=137$ \alert{(Witness)}\\
%
(1) $\textrm{mcd}(n,b) \neq 1$ & $\textrm{mcd}(221,137)=1$ \\
%
(2) $b^m \bmod n \neq 1$ \AND\ & $137^{55} \bmod 221 = 188 \neq 1$\\
\phantom{(2)} $\forall i, 0 \leq i < v: b^{m \cdot 2^i} \bmod n \neq n-1$ &
$137^{55 \cdot 2^0} \bmod 221 = 188 \neq 220$ \\
%
& $137^{55 \cdot 2^1} \bmod 221 = 205 \neq 220$ 
\end{tabular}
\end{myboxtitle}

\end{frame}


%-------------------------------------------------------------------------
\begin{frame}{Test di primalità -- Miller-Rabin}


\vspace{-9pt}
\begin{myboxtitle}[Verifica primalità $n$ -- Esempio: $n=229$ numero primo]
\small
\begin{tabular}{m{6cm}m{5cm}}
Sia $n$ un numero dispari & $n=229$ \\
%
Sia $n-1 = m \cdot 2^v$, con $m$ dispari & $228 = 57 \cdot 2^2$ \\
%
$n-1$ in binario è pari a:\newline 
$m$ in binario seguito da $v$ zeri & $\overbrace{111001}^{m=57}\underbrace{00}_{v=2}$\\
%
se $\exists b, 2 \leq b < n$ per cui una delle seguenti affermazioni è vera, $n$ è \alert{composto}: & Proviamo $b=137$ \\
%
(1) $\textrm{mcd}(n,b) \neq 1$ & $\textrm{mcd}(229,137)=1$ \\
%
(2) $b^m \bmod n \neq 1$ \AND\ & $137^{57} \bmod 229 = 188 \neq 1$\\
\phantom{(2)} $\forall i, 0 \leq i < v: b^{m \cdot 2^i} \bmod n \neq n-1$ &
$137^{57 \cdot 2^0} \bmod 229 = 122 \neq 228$ \\
%
& $137^{57 \cdot 2^1} \bmod 229 = \alert{228}$ 
\end{tabular}
\end{myboxtitle}

\end{frame}

%-------------------------------------------------------------------------
\begin{frame}{Test di primalità -- Miller-Rabin}


\vspace{-9pt}
\begin{myboxtitle}[Verifica primalità $n$ -- Esempio: $n=221 = 13 \times 17$]
\small
\begin{tabular}{m{6cm}m{5cm}}
Sia $n$ un numero dispari & $n=221$ \\
%
Sia $n-1 = m \cdot 2^v$, con $m$ dispari & $220 = 55 \cdot 2^2$ \\
%
$n-1$ in binario è pari a:\newline 
$m$ in binario seguito da $v$ zeri & $\overbrace{110111}^{m=55}\underbrace{00}_{v=2}$\\
%
se $\exists b, 2 \leq b < n$ per cui una delle seguenti affermazioni è vera, $n$ è \alert{composto}: & $b=174$ (\alert{Strong liar})\\
%
(1) $\textrm{mcd}(n,b) \neq 1$ & $\textrm{mcd}(221,174)=1$ \\
%
(2) $b^m \bmod n \neq 1$ \AND\ & $174^{55} \bmod 221 = 47 \neq 1$\\
\phantom{(2)} $\forall i, 0 \leq i < v: b^{m \cdot 2^i} \bmod n \neq n-1$ &
$174^{55 \cdot 2^0} \bmod 221 = 47 \neq 220$ \\
%
& $174^{55 \cdot 2^1} \bmod 221 = \alert{220}$ 
\end{tabular}
\end{myboxtitle}

\end{frame}


%-------------------------------------------------------------------------
\begin{frame}{Test di primalità -- Miller-Rabin}

\vspace{-9pt}
\BIL
\item Rabin ha dimostrato che se $n$ è composto, allora ci sono
almeno $\sfrac{3}{4}\,(n-1)$ testimoni in $[2, \ldots, n-1]$ 
\item Il test di compostezza ha una probabilità inferiore a $1/4$ di rispondere
erroneamente
\EIL

\begin{Procedure}
\caption[A]{\BOOLEAN\ \textsf{isPrime}(\INTEGER\ $n$)}
\For{$i=1$ \TO\ $k$}{
  $b = \textsf{random}(2, n-1)$\;
  \If{$\textsf{isComposite}(n,b)$}{
    \Return \FALSE\;
  }
}
\Return \TRUE\;
\end{Procedure}

\end{frame}

%-------------------------------------------------------------------------
\begin{frame}{Test di Miller-Rabin}

\vspace{-9pt}
\begin{myboxtitle}[Riassunto]
\BIL
\item Complessità: $O(k \log^2 n \log \log n \log \log \log n)$
\item Probabilità di errore: $(1/4)^k$
\item Algoritmo di tipo Montecarlo
\EIL
\end{myboxtitle}

\begin{myboxtitle}[Algoritmi probabilistici vs algoritmi deterministici]
\BIL
\item Dal 2002, esiste l'algoritmo deterministico AKS di complessità $O(\log^{6+\epsilon} n)$
\item I fattori moltiplicativi coinvolti sono molto alti
\item Si preferisce quindi l'algoritmo di Miller-Rabin
\EIL
\end{myboxtitle}

\end{frame}

\subsection{Espressione polinomiale nulla}

%-------------------------------------------------------------------------
\begin{frame}{Esempio -- Espressione polinomiale nulla}

\vspace{-9pt}
\begin{myboxtitle}[Problema]
Data un’espressione algebrica polinomiale $p(x_1 , \ldots , x_n)$ 
in $n$ variabili, determinare se $p$ è identicamente nulla oppure no.
\end{myboxtitle}

\begin{myboxtitle}[Discussione]
\BIL
\item Assumiamo che non sia in forma di monomi - altrimenti è banale
\item Gli algoritmi basati su semplificazioni sono molto complessi
\EIL
\end{myboxtitle}

\end{frame}


%-------------------------------------------------------------------------
\begin{frame}{Esempio -- Espressione polinomiale nulla}

\vspace{-9pt}
\begin{myboxtitle}[Algoritmo]
\BIL
\item Si genera una $n$-pla di valori $v_1, \ldots, v_n$
\item Si calcola $x= p(v_1 , \ldots , v_n)$
  \BI
	\item Se $x \neq 0$, $p$ non è identicamente nulla
	\item Se $x = 0$, $p$ \alert{potrebbe} essere identicamente nulla
	\EI
\item Se ogni $v_i$ è un valore intero compreso casuale fra $1$ e $2d$, dove $d$ è 
il grado del polinomio, allora la probabilità di errore non supera $1/2$.
\item Si ripete $k$ volte, riducendo la probabilità di errore a $(1/2)^k$
\EIL
\end{myboxtitle}

\end{frame}

\subsection{Strutture dati probablistiche}


\begin{frame}{BitSet + Tabelle Hash = Bloom Filters}

\vspace{-12pt}
\begin{columns}[c]
\begin{column}{0.48\textwidth}
\begin{myboxtitle}[BitSet]
\alert{Vantaggi}
\BI
\item 1 bit/oggetto
\EI
\medskip
\alert{Svantaggi}
\BI
\item Elenco prefissato di oggetti
\EI
\end{myboxtitle}

\end{column}
\hfill
\begin{column}{0.48\textwidth}
\begin{myboxtitle}[Tabelle Hash]
\alert{Vantaggi}
\BI
\item Struttura dati dinamica
\EI
\medskip
\alert{Svantaggi}
\BI
\item Alta occupazione di memoria
\EI
\end{myboxtitle}

\end{column}
\end{columns}

\medskip
\begin{myboxtitle}[Bloom filters]
\medskip
\TwoCols{
\alert{Vantaggi}
\BI
\item Struttura dati dinamica
\item Bassa occupazione di memoria
(10 bit/oggetto)
\EI
}
{
\alert{Svantaggi}
\BI
\item Niente cancellazioni
\item Risposta probabilistica
\item No memorizzazione
\EI
}
\end{myboxtitle}
\end{frame}

%-------------------------------------------------------------------------
\begin{frame}{Bloom filter}

\vspace{-6pt}
\begin{myboxtitle}[Specifica]
\BI
\item $\textsf{insert}(\mathit{key})$: Inserisce l'elemento $\mathit{key}$ nel bloom filter
\item $\BOOLEAN\ \textsf{contains}(\mathit{key})$\\
\BI
\item Se restituisce \FALSE, l'elemento $\mathit{key}$ è sicuramente non presente
nell'insieme
\item Se restituisce \TRUE, l'elemento $\mathit{key}$ può essere presente oppure no
(\alert{falsi positivi})
\EI
\EI
\end{myboxtitle}

\end{frame}

%-------------------------------------------------------------------------
\begin{frame}{Bloom filter}

\vspace{-6pt}
\BB{Trade-off fra occupazione di memoria e probabilità di falso positivo}
\TwoColsCustom{0.7}{0.26}{
\BI
\item Sia $\epsilon$ la probabilità di falso positivo
\item I bloom filter richiedono $1.44\log_2(1/\epsilon)$ bit per elemento
inserito
\EI
}{
\begin{tabular}{|c|r|}
\hline
$\epsilon$ & Bit \\
\hline
$10^{-1}$ & 4.78 \\
\hline
$10^{-2}$ & 9.57 \\
\hline
$10^{-3}$ & 14.35 \\
\hline
$10^{-4}$ & 19.13 \\
\hline
\end{tabular}
}
\end{frame}


\begin{frame}{Applicazioni dei Bloom Filter}
	
\BB{Chrome Safe Browsing}
\BIL
\item Chrome contiene un database delle URL associate a siti con malware, costantemente aggiornato
\item Fino al 2012, memorizzato con un Bloom Filter
\item Chrome verifica l'appartenenza di ogni URL al database
\BI
\item Se la risposta è \FALSE, non appartiene
\item Se la risposta è \TRUE, potrebbe appartenere e viene fatta una verifica
tramite un servizio centralizzato di Google
\EI
\EIL

\BB{Qualche dato (da prendere cum grano salis)}
\BI
\item Nel 2011, 650k URL memorizzati in 1.94MB
\item 25 bit per URL, $\epsilon \approx 10^{-5}$
\EI

\end{frame}

\begin{frame}{Applicazioni dei Bloom Filter}

\BB{Ogni qual volta una verifica locale permette di evitare un'operazione
più costosa, quali operazioni di I/O e comunicazioni di rete}

\begin{myboxtitle}[They say...]
\BI
\item \alert{Medium} uses Bloom filters to avoid recommending articles a user has previously read.
\item \alert{Apache HBase} uses bloom filter to boost read speed by filtering out unnecessary disk reads of HFile blocks which do not contain a particular row or column.
\item \alert{Ethereum} uses Bloom filters for quickly finding logs on the Ethereum blockchain.
\EI
\bigskip
\tiny{
https://en.wikipedia.org/wiki/Bloom\_filter
}
\end{myboxtitle}

	
\end{frame}

\begin{frame}{Applicazioni dei Bloom Filter}

\begin{myboxtitle}[Firefox CRLite]
CRLite is a technology proposed by a group of researchers at the IEEE Symposium on Security and Privacy 2017 that compresses revocation information so effectively that 6.7 GB of revocation data can become 1.3 MB. It accomplishes this by combining Certificate Transparency data and Internet scan results with cascading Bloom filters, building a data structure that is reliable, easy to verify, and easy to update.

\bigskip
\tiny{
https://blog.mozilla.org/security/2020/01/09/crlite-part-1-all-web-pki-revocations-compressed/ \\
https://blog.mozilla.org/security/2020/01/09/crlite-part-2-end-to-end-design/
}

\end{myboxtitle}
	
\end{frame}



%-------------------------------------------------------------------------
\begin{frame}{Implementazione}

\vspace{-6pt}
\begin{overprint}
\onslide<1|handout:1>
\BI
\item Un vettore booleano $A$ di $m$ bit, inizializzato a \FALSE 
\item $k$ funzioni hash $h_1, h_2, \ldots, h_k: U \rightarrow [0, m-1]$
\EI
\onslide<2-4|handout:2>
\vspace{-12pt}
\begin{Procedure}
\caption[A]{$\textsf{insert}(\mathit{key})$}
\For{$i = 1$ \TO\ $k$}{
  $A[h_i(\mathit{key})] = \TRUE$\;
}
\end{Procedure}
\onslide<5|handout:3>
\vspace{-12pt}
\begin{Procedure}
\caption[A]{$\BOOLEAN\ \textsf{contains}(\mathit{key})$}
\For{$i = 1$ \TO\ $k$}{
  \lIf{$A[h_i(\mathit{key})] \Eq \FALSE$}{
		\Return \FALSE
	}
}
\Return \TRUE\;
\end{Procedure}
\end{overprint}

\vspace{-6pt}
\begin{overprint}
\includegraphics<1|handout:1>[width=1.0\textwidth,page=1]{bloom-crop.pdf}
\includegraphics<2|handout:0>[width=1.0\textwidth,page=2]{bloom-crop.pdf}
\includegraphics<3|handout:0>[width=1.0\textwidth,page=3]{bloom-crop.pdf}
\includegraphics<4|handout:2>[width=1.0\textwidth,page=4]{bloom-crop.pdf}
\includegraphics<5|handout:3>[width=1.0\textwidth,page=5]{bloom-crop.pdf}
\end{overprint}

\end{frame}

\begin{frame}{Qualche formula (senza dimostrazione)}
\BI
\item Dati $n$ oggetti, $m$ bit, $k$ funzioni hash, la probabilità di un falso
positivo è pari a:

\[
 \epsilon = \left( 1 - e^{-kn/m} \right)^k
\]
\item Dati $n$ oggetti e $m$ bit, il valore ottimale per $k$ è pari a

\[
  k = \frac{m}{n}\ln 2
\]
\item Dati $n$ oggetti e una probabilità di falsi positivi $\epsilon$, il numero
di bit $m$ richiesti è pari a:

\[
 m = -\frac{n \ln \epsilon}{(\ln 2)^2}
\]
\EI

\end{frame}


\section{Algoritmi Las Vegas}

\subsection{Problema della selezione}

%-------------------------------------------------------------------------
\begin{frame}{Statistica}

\vspace{-9pt}
\begin{myboxtitle}[Algoritmi statistici su vettori]
Estraggono alcune caratteristiche statisticamente rilevanti da un vettore numerico 
\end{myboxtitle}

\begin{myboxtitle}[Esempi]
\BIL
\item \alert{Media}: $\mu = \frac{1}{n} 	\sum_{i=1}^n A[i]$
\item \alert{Varianza}: $\sigma^2 = \frac{1}{n} \sum_{i=1}^n (A[i]- \mu)^2$
\item \alert{Moda}: il valore (o i valori) più frequenti
\EIL
\end{myboxtitle}
\end{frame}

%-------------------------------------------------------------------------
\begin{frame}{Statistiche d'ordine}

\vspace{-9pt}
\begin{myboxtitle}[Selezione]
Dato un array $A$ contenente $n$ valori e un valore $1 \leq k \leq n$, 
trovare l'elemento che occuperebbe la posizione $k$ se il vettore fosse ordinato
\end{myboxtitle}

\begin{myboxtitle}[Mediana]
Il problema del calcolo della mediana è un sottoproblema del problema
della selezione con $k=\lceil n/2 \rceil$.
\end{myboxtitle}

\BB{Come risolvereste il problema?}
\pause
Ovviamente, possiamo ordinare i valori e andare a cercare il valore in posizione $k$, con costo $O(n \log n)$. Possiamo fare meglio di così?

\end{frame}

%-------------------------------------------------------------------------
\begin{frame}{Selezione per piccoli valori di $k$}

\vspace{-9pt}
\begin{myboxtitle}[Intuizione]
\BIL
\item Si può utilizzare uno heap
\item L'algoritmo può essere generalizzato a valori generici di $k > 2$
\EIL
\end{myboxtitle}

\TwoColsCustom{0.59}{0.40}{
\begin{Procedure}
\caption[A]{\INTEGER \textsf{heapSelect}(\Item[\,]\ $A$, \INTEGER $n$, \INTEGER $k$)}
  $\textsf{buildHeap}(A)$\;
  \For{$i = 1$ \TO\ $k-1$}{
    $\textsf{deleteMin}(A, n)$\;
  }
  \Return $\textsf{deleteMin}(A,n)$\;
\end{Procedure}
}{

\begin{myboxtitle}[Complessità]
\BIL
\item $O(n + k \log n)$
\item Se $k = O(n/\log n)$,\\ il costo è $O(n)$
\item Se $k = n/2$, non va bene
\EIL
\end{myboxtitle}
}
\end{frame}
	
%-------------------------------------------------------------------------
\begin{frame}{Idea}

\vspace{-9pt}
\BIL
\item Approccio divide-et-impera simile al Quicksort
\item Essendo un problema di ricerca, non è necessario cercare in entrambe le partizioni, basta cercare in una sola di esse
\item Bisogna fare attenzione agli indici
\EIL

\end{frame}

%-------------------------------------------------------------------------
\begin{frame}{Algoritmo di selezione}

\vspace{-12pt}
\small
\begin{Procedure}
\caption[A]{\Item \textsf{selection}($\Item[\,]\ A$, \INTEGER $\Primo$, \INTEGER $\Ultimo$, \INTEGER $k$)}
\eIf{$\Primo \Eq \Ultimo$}{
  \Return $A[\Primo]$\;
}{
  $\INTEGER\ j = \textsf{pivot}(A, \Primo, \Ultimo)$\;
  $\INTEGER\ q = j - \Primo + 1$\;
  \uIf{$k \Eq q$}{
    \Return $A[j]$\;
  }
  \uElseIf{$k<q$}{
    \Return $\textsf{selection}(A, \Primo, j-1, k)$\;
  }
  \Else{
    \Return $\textsf{selection}(A, j+1, \Ultimo, k-q)$\;
  }
}
\end{Procedure}
\vspace{-18pt}
\IG{0.8}{mediana-crop.pdf}

\end{frame}

%-------------------------------------------------------------------------
\begin{frame}{Complessità}

\vspace{-9pt}
\begin{myboxtitle}[Caso pessimo]
\begin{align*}
  T(n) &= \begin{cases}
    1 & n \leq 1 \\
    T(n-1) + n & n>1 \\
  \end{cases} \\
  T(n) &= O(n^2)
\end{align*}  
\end{myboxtitle}

\begin{myboxtitle}[Caso ottimo]
\begin{align*}
  T(n) &= \begin{cases}
    1 & n \leq 1 \\
    T(n/2) + n & n>1 \\
  \end{cases} \\
  T(n) &= O(n)
\end{align*}  
\end{myboxtitle}




\end{frame}

%-------------------------------------------------------------------------
\begin{frame}{Complessità}

\vspace{-9pt}
\begin{myboxtitle}[Caso medio]
Assumiamo che \textsf{pivot}() restituisca con la stessa probabilità una qualsiasi posizione $j$ del vettore $A$
\begin{align*}
T(n) &= n + \frac{1}{n}\sum_{q=1}^{n} T\left(\max\{q-1, n-q\}\right)&&\textrm{Media su $n$  casi}\\
     &\le n + \frac{1}{n} \sum_{q=\lfloor n/2\rfloor}^{n-1} 2T(q), &&\textrm{per $n > 1$}
\end{align*}
\end{myboxtitle}

\begin{myboxtitle}[Esempi ]
\small
\BIL
\item $n=4$:
$\max\{0, \alert{3} \} +
\max\{1, \alert{2} \} +
\max\{\alert{2}, 1 \} +
\max\{\alert{3}, 0 \}$
\item $n=5$:
$\max\{0, \alert{4} \}+
\max\{1, \alert{3} \}+
\max\{2, \alert{2} \}+
\max\{\alert{3}, 1 \}+
\max\{\alert{4}, 0 \}$
\EIL
\end{myboxtitle}

\end{frame}

%-------------------------------------------------------------------------
\begin{frame}[shrink=5]{Complessità}

% \vspace{-9pt}
% \begin{align*}
% T\left(n\right) &\le n - 1 + \frac{2c}{n} \sum_{q=\left\lfloor n/2\right\rfloor}^{n-1}q \\
% &\le n + \frac{2c}{n}\left(\sum_{q=1}^{n-1}q - \sum_{q=1}^{\left\lfloor n/2\right\rfloor-1}q\right)\\
% &= n + \frac{2c}{n}\left(\frac{n\left(n- 1\right)}{2} - \frac{\left\lfloor n/2\right\rfloor\left(\left\lfloor n/2\right\rfloor - 1\right)}{2}\right)\\
% &\le n + \frac{c}{n}\big(n\left(n- 1\right) - \left(n/2+1\right)\left(n/2\right)\big)\\
% &= n + c\left(n - 1\right) - \left(c/2\right)\left(n/2+ 1\right) = n + cn - c - cn/4 - c/2\\
% &= cn \left( \frac{1}{c}+ \frac{3}{4} - \frac{3}{2n}\right) \leq cn \left( \frac{1}{c}+ \frac{3}{4}\right) \leq cn
% \end{align*}



\vspace{-15pt}
\small
\begin{align*}
T\left(n\right) &\le 
n + \frac{1}{n} \sum_{q=\left\lfloor \sfrac{n}{2}\right\rfloor}^{n-1} 2 \cdot cq \leq
n + \frac{2c}{n} \sum_{q=\left\lfloor \sfrac{n}{2}\right\rfloor}^{n-1}q
&&\textrm{Sostituzione, raccolgo $2c$} \\
&= n + \frac{2c}{n} \left( \sum_{q=1}^{n-1}q - \sum_{q=1}^{\lfloor \sfrac{n}{2} \rfloor -1} q \right)&&\textrm{Sottrazione prima parte}\\
&= n + \frac{2c}{n} \cdot \left(\frac{n\left(n- 1\right)}{2} - \frac{\left(\lfloor \sfrac{n}{2}\rfloor\right)\left(\lfloor\sfrac{n}{2}\rfloor - 1\right)}{2}\right)\\
&\leq n + \frac{2c}{n} \cdot \left(\frac{n\left(n- 1\right)}{2} - \frac{\left(\sfrac{n}{2} -1\right)\left(\sfrac{n}{2} - 2\right)}{2}\right)&&\textrm{Rimozione limite inferiore}\\
&= n + \frac{\cancel{2}c}{n} \cdot \frac{\left(n^2-n - \left(\sfrac{1}{4}\,n^2 - \sfrac{3}{2}\,n + 2\right)\right)}{\cancel{2}} \\
&= n + \sfrac{c}{n} \cdot \left(\sfrac{3}{4}\,n^2+ \sfrac{1}{2}\,n-2 \right) \\
&\leq n + \sfrac{c}{n} \cdot \left(\sfrac{3}{4}\,n^2 + \sfrac{1}{2}\,n\right) = n + \sfrac{3}{4}\,cn + \sfrac{1}{2} \stackrel{?}{\leq} cn&& \textrm{Vera per $c \geq 6$ e $n \geq 1$}\\
\end{align*}


\end{frame}

%-------------------------------------------------------------------------
\begin{frame}{Complessità}
   
\vspace{-9pt}
\BIL
\item  Siamo partiti dall’assunzione 
  \BI
  \item $j$ assume equiprobabilisticamente tutti i valori compresi fra $1$ e $n$
  \EI
\item E se non fosse vero?
\item Lo forziamo noi!
  \BI
  \item  $A[\textsf{random}(\Primo, \Ultimo)] \leftrightarrow A[\Primo]$
  \EI
\item Questo accorgimento vale anche per QuickSort    
\item La complessità nel caso medio:
  \BI
  \item $O(n)$ nel caso della Selezione
  \item $O(n \log n)$ nel caso dell'Ordinamento
  \EI
\EIL

\end{frame}

%-------------------------------------------------------------------------
\begin{frame}{Selezione deterministica}

\vspace{-9pt}
\begin{myboxtitle}[Algoritmo black-box]
Supponiamo di avere un algoritmo “black box” che mi ritorni il mediano di $n$ valori in tempo $O(n)$
\end{myboxtitle}

\begin{myboxtitle}[Domande]
\BIL
\item Potrei utilizzarlo per ottimizzare il problema della selezione?
\item Che complessità otterrei?
\EIL
\end{myboxtitle}

\end{frame}

%-------------------------------------------------------------------------
\begin{frame}{Selezione deterministica}

\vspace{-9pt}
\begin{myboxtitle}[Se conoscessi tale algoritmo]
\BIL
\item il problema della selezione sarebbe quindi risolto...
\item ... ma dove lo trovo un simile algoritmo?
\EIL
\end{myboxtitle}

\begin{myboxtitle}[Rilassiamo le nostre pretese]
\BIL
\item Supponiamo di avere un algoritmo “black box” che mi ritorni un valore che dista al più $\frac{3}{10}n$ dal mediano (nell'ordinamento)
\item Potrei utilizzarlo per ottimizzare il problema della selezione?
\item Che complessità otterrei?
\EIL
\end{myboxtitle}

\end{frame}

\begin{frame}{Selezione deterministica}

\vspace{-9pt}
\begin{myboxtitle}[Idea]
\BIL
\item Suddividi i valori in gruppi di 5. Chiameremo l'$i$-esimo gruppo $S_i$, con $i \in \left[1, \lceil n/5 \rceil \right]$
\item Trova il mediano $M_i$ di ogni gruppo $S_i$
\item Tramite una chiamata ricorsiva, trova il mediano $m$ delle mediane
$[M_1, M_2, \ldots, M_{\lceil n/5 \rceil}]$
\item Usa $m$ come pivot e richiama l'algoritmo ricorsivamente sull'array opportuno, come nella \textsf{selection}() randomizzata
\item Quando la dimensione scende sotto una certa dimensione, possiamo
utilizzare un algoritmo di ordinamento per trovare il mediano
\EIL
\end{myboxtitle}

\end{frame}

\begin{frame}{Selezione deterministica}

\vspace{-12pt}
\small
\begin{Procedure}
\caption[A]{\Item\ \textsf{select}($\Item[\,]\ A$, \INTEGER $\Primo$, \INTEGER $\Ultimo$, \INTEGER $k$)}
\% Se la dimensione è inferiore ad una soglia (10), ordina il vettore e\;
\% restituisci il $k$-esimo  elemento di $A[\Primo \ldots \Ultimo]$\;
\If{$\Ultimo-\Primo+1 \leq 10$}{
  $\textsf{InsertionSort}(A, \Primo, \Ultimo)$\Comment*[f]{Versione con indici inizio/fine}\;
  \Return $A[\Primo+k-1]$\;
}
\BlankLine
\% Divide $A$ in $\lceil n/5 \rceil$ sottovettori di dim. $5$ e ne calcola la mediana\;
$M = \NEW\ \INTEGER[1 \ldots \lceil n/5 \rceil]$\;
\For{$i = 1$ \TO $\lceil n/5 \rceil$}{
  $M[i] = \textsf{median5}(A, \Primo+(i-1) \cdot 5, \Ultimo)$\;
}
\BlankLine
\% Individua la mediana delle mediane e usala come perno\;
$\Item\ m = \textsf{select}(M, 1, \lceil n/5 \rceil, \lceil \lceil n/5 \rceil/2 \rceil)$\;
$\INTEGER\ j = \textsf{pivot}(A, \Primo, \Ultimo, m)$\Comment*[f]{Versione con $m$ in input}\;
[...]\;
\end{Procedure}

\end{frame}

\begin{frame}{Selezione deterministica}

\vspace{-12pt}
\small
\begin{Procedure}
\caption[A]{\Item\ \textsf{select}($\Item[\,]\ A$, \INTEGER $\Primo$, \INTEGER $\Ultimo$, \INTEGER $k$)}
[...]\;

\BlankLine
\% Calcola l'indice $q$ di $m$ in $[\Primo \ldots \Ultimo]$\;
\% Confronta $q$ con l'indice cercato e ritorna il valore conseguente\;
$\INTEGER\ q = j-\Primo+1$\;
\uIf{$q \Eq k$}{
  \Return $m$\;
}
\uElseIf{$q < k$}{
  \Return $\textsf{select}(A, \Primo, q-1, k)$\;
}
\Else{
  \Return $\textsf{select}(A, q+1, \Ultimo, k-q)$\;
}
\end{Procedure}

\end{frame}


\begin{frame}{Selezione deterministica}

\vspace{-9pt}
\BIL
\item  Il calcolo dei mediani $M[\,]$ richiede al più $6 \lceil n/5 \rceil$  confronti.
\item La prima chiamata ricorsiva dell'algoritmo $\textsf{select}()$ viene effettuata su $\lceil n/5 \rceil$ elementi
\item La seconda chiamata ricorsiva dell'algoritmo $\textsf{select}()$ viene effettuata al massimo  su $7n/10$ elementi\\
 (esattamente $n - 3 \lceil  \lceil n/5 \rceil /2 \rceil$) 
\item L'algoritmo $\textsf{select}()$ esegue nel caso pessimo $O(n)$ confronti

\[
  T(n) = T(n/5) + T(7n/10) + 11/5n
\]
\EIL

\begin{center}
\begin{tabular}{|c|c|c|c|c||c|c|c|c|c||c|c|c|c|c|}
\hline
\alert{1} & \alert{4} & \alert{\textbf{7}} & 10 & 13 & \alert{2} & \alert{5} & \alert{\textbf{8}} & 11 & 14 & 3 & 6 & \textbf{9} & 12 & 15 \\\hline
\end{tabular}
\end{center}

\end{frame}

\begin{frame}{Conclusioni}

\vspace{-9pt}
\begin{myboxtitle}[Quale preferire?]
\BIL
\item Algoritmo probabilistico Las Vegas in tempo atteso $O(n)$
\item Algoritmo deterministico in tempo $O(n)$, con fattori moltiplicativi
  più alti
\EIL
\end{myboxtitle}


\begin{myboxtitle}[Note storiche]
\BIL
\item Nel 1883 Lewis Carroll (!) notò che il secondo premio nei tornei di tennis non veniva assegnato in maniera equa.
\item Nel 1932, Schreier dimostrò che $n + \log n - 2$  incontri sono sempre sufficienti per trovare il secondo posto
\item Nel 1973, a opera di Blum, Floyd, Pratt, Rivest e Tarjan, appare il primo algoritmo deterministico
\EIL
\end{myboxtitle}
\end{frame}

\end{document}





